\documentclass[letterpaper,12pt]{article}

% Import packages from .sty file.
\usepackage{imports}


%%%%%%%%%%%%%%%%%%%%%%%%%%%%%%%%%%%%%%%%%%%%%%%%%%%%%%%%%%%%%%%%%%
%%%%%%%%%%%%%%%%%%%%%%%%%%%%%%%%%%%%%%%%%%%%%%%%%%%%%%%%%%%%%%%%%%

% Proposal for Tanner idea1/Mike idea4

%%%%%%%%%%%%%%%%%%%%%%%%%%%%%%%%%%%%%%%%%%%%%%%%%%%%%%%%%%%%%%%%%%
%%%%%%%%%%%%%%%%%%%%%%%%%%%%%%%%%%%%%%%%%%%%%%%%%%%%%%%%%%%%%%%%%%


% Set up the title.
\title{Can We Measure Teacher Differentiation? }
\author{Tanner S Eastmond\thanks{Department of Economics, University of California, San Diego. Email: teastmond@ucsd.edu},  Nathan Mather\thanks{Department of Economics, University of Michigan. Email: njmather@umich.edu }, Michael Ricks\thanks{Department of Economics, University of Michigan. Email: ricksmi@umich.edu}}
\date{\vspace{-8ex}}

\begin{document}
\maketitle


\section{Question:}
What can we learn about a teacher’s skill at differentiation by measuring her student’s test score gains across the achievement distribution?

\section{Motivation and Proposal:}
Differentiation is one of the most crucial teacher skills, and it is almost impossible to measure. The current VAMs don't get at it, and Tanner knows of no good measure/approximation to think about this skill empirically.

We propose constructing a measure of class composition (e.g., is it a highly skilled class, low skilled, high variance in ability, etc.) using prior year test scores. This measure could be percent of students who are above some threshold in the achievement distribution, below some threshold, the variance of ability in the class, or some mixture of these. With these measures we would use a quantile regression to estimate the gains for students across the achievement distribution. After we have these estimates, we can compare them across years for each teacher. Teachers who show little variance in how well their high achieving students perform, for example, despite different class compositions, are those better at differentiating. Another way to address the question is to see if our class composition measure has a strong association with scores for students in each decile of the achievement distribution. If it is, we can then get the predicted values from each regression and find teachers whose students’ scores across the distribution differ significantly from the predictions, and those would be teachers who were better (or worse) at differentiating, particularly if those teachers’ scores are different every year.

\section{The Econometrics:}



\section{Notes:}

Peer effects might ruin this one. Can we tell differentiation from peer effects?

Idea is  unique in a busy literature and could provide a strong contribution: a whole new measure of quality

Methodological innovation is not huge. But it is new take on measuring something we care about. Policy implications about what ``good" teachers really are ``good" at. This has implications for the policy ideas about moving around or firing teachers. Are the best teachers also good at helping low students? Should we move them? If so which?

Differentiation could also be measured with two different groups (variation across classes within teacher---not just high variance class) Both are interesting.

Sample sizes are small. This is always a problem with VA, so this will work best in subgroups that teach a lot of students.




\end{document}
