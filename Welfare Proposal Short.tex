\documentclass[letterpaper,12pt]{article}

% Import packages from .sty file.
%\usepackage{imports}

% Mike's things because he couldn't figure out how to get the preamble working otherwise
\usepackage[utf8]{inputenc}
\usepackage{geometry,ulem,graphicx,caption,color,setspace,dsfont,amssymb}
\usepackage{natbib}
\usepackage{subcaption} 
\usepackage[short]{optidef}
\usepackage{hhline}
\usepackage[capposition=top]{floatrow}
\usepackage{booktabs} % Allows the use of \toprule, \midrule and \bottomrule in tables
\usepackage{adjustbox}
\usepackage{tikz}
\usepackage{pdflscape}
\usetikzlibrary{calc,patterns,positioning}
\usepackage{environ}





% commands for use to put comments into text 

\newcommand\cmnt[2]{\;
{\textcolor{red}{[{\em #1 --- #2}] \;}
}}

\newcommand\Nate[1]{\cmnt{#1}{Nate}}
\newcommand\Mike[1]{\cmnt{#1}{Mike}}
\newcommand\Tanner[1]{\cmnt{#1}{Tanner}}
\newcommand\rmk[1]{\;\textcolor{red}{{\em #1}\;}}
\newcommand\Natenote[1]{\footnote{\cmnt{#1}{Nate}}}
\newcommand\Mikenote[1]{\footnote{\cmnt{#1}{Mike}}}
\newcommand\Tannernote[1]{\footnote{\cmnt{#1}{Tanner}}}


\usepackage[utf8]{inputenc}
\usepackage[english]{babel}
 
\usepackage{natbib}
\bibliographystyle{apa}

\setcitestyle{round}
\setcitestyle{semicolon}
\setcitestyle{yysep={;}}

\usepackage{titlesec}
\titleformat{\section}
  {\normalfont\normalsize\bfseries}{\thesection.}{1em}{}

\titleformat{\subsection}
  {\normalfont\normalsize\bfseries}{\thesubsection}{1em}{}


%%%%%%%%%%%%%%%%%%%%%%%%%%%%%%%%%%%%%%%%%%%%%%%%%%%%%%%%%%%%%%%%%%
%%%%%%%%%%%%%%%%%%%%%%%%%%%%%%%%%%%%%%%%%%%%%%%%%%%%%%%%%%%%%%%%%%

% Proposal for Tanner idea4/Mike idea1

%%%%%%%%%%%%%%%%%%%%%%%%%%%%%%%%%%%%%%%%%%%%%%%%%%%%%%%%%%%%%%%%%%
%%%%%%%%%%%%%%%%%%%%%%%%%%%%%%%%%%%%%%%%%%%%%%%%%%%%%%%%%%%%%%%%%%


% Set up the title.
\title{From Value Added to Welfare Added: A Social Planner Approach to Education Policy and Statistics}
\author{
Tanner S Eastmond\thanks{Department of Economics, University of California, San Diego. Email: teastmon@ucsd.edu}, Nathan Mather\thanks{Department of Economics, University of Michigan. Email: njmather@umich.edu }, Michael Ricks\thanks{Department of Economics, University of Michigan. Email: ricksmi@umich.edu}} 
\date{\vspace{-8ex}}

\begin{document}
%\maketitle
\begin{center}
\noindent \textbf{From Value Added to Welfare Added: A Social Planner Approach to Education Policy and Statistics}

Tanner S Eastmond, Nathan Mather, Michael Ricks
\end{center}



%%%%%%%%%%%%%%%%%%%%%%%%%%%%%%%%%%%%%%%%%%%%%%%%%%%%%%%%
%%%%%%%%%%%%%%% Motivation and Proposal. %%%%%%%%%%%%%%%
%%%%%%%%%%%%%%%%%%%%%%%%%%%%%%%%%%%%%%%%%%%%%%%%%%%%%%%%
\section{Motivation and Proposal:}
A major goal of value added measures (VAM) is to determine who are the most and least effective educators and schools. Despite criticisms, research has shown the use of Value Added Measures increases  teacher (and therefore student) performance and capture teachers' long term impacts on students \citep{chetty2014measuring2, pope2019effect}. 

While these results support the increasing reliance on Value Added Measures of teacher performance, there seems to be a philosophical disconnect between traditional  VAM, which measures mean performance, and education policy such as No Child Left Behind (NCLB), which attempted to target low performing students. The heart of this tension lies in the fact that VAM are mean-oriented statistics; VAM represents a teachers average value imparted to their students. While making them much more empirically tractable, this also limits the breadth of questions VAM can address and what policy ideas they can be reasonably used to promote. 

How do teachers vary in their abilities to help different types of students? How well does ranking teachers using VAM reflect those differences? How much do teachers' rankings change under different policies' ``welfare'' criteria?  What implications do these changes have for the effectiveness of educational accountability policies? Because Traditional VAM  struggle to address these questions, we propose a set of more flexible heterogeneous VAM that measure teachers heterogeneous impacts and weight that impact on various students according to a given policy goal or welfare criterion. 

The three methods we propose for estimating heterogeneous VAM are a weighted least squares regression of residuals, empirical Bayes estimates, and quantile regressions. Regardless of the exact method used, the end result will be value added measures that can be adapted to a specific normative policy goal. For example, No Child Left Behind was intended to bring all students up to a certain level and to hold schools accountable for groups that were traditionally overlooked. While traditional value added measures only capture the average effect of a teacher, our heterogeneous value added measures could weigh performance gains to lower achieving students more than those to higher performing students. This more felxible VAM would enable policy makers to align teacher incentives to the stated policy goal.\footnote{For example in the NCLB situation, the current evaluation structure meant that students at the bottom and top of the achievement distribution were often overlooked \citep{neal2010left}. But this could be over come by a continuous weighting structure rather than a discontinuous cutoff.} 

While all of our potential approaches to estimating heterogeneous VAM should perform well in large samples, we will need to show how they perform in controlled/simulated environments to make sure that our estimators can recover true parameters in reasonable empirical settings. Once we have confidence in the estimators from simulations, we will be able to calculate these scores for real teachers then rank them under various policy welfare criteria. This will demonstrate the effectiveness of these metrics on a real distribution of students and teachers.  We would also like to consider how teachers respond to incentives, simulate how they reoptimize, and show the resulting changes in the distribution of effects.


%%%%%%%%%%%%%%%%%%%%%%%%%%%%%%%%%%%%%%%%%%%%%%%%%%%%%%%%
%%%%%%%%%%%%%%%%%%%%%% Estimation %%%%%%%%%%%%%%%%%%%%%%
%%%%%%%%%%%%%%%%%%%%%%%%%%%%%%%%%%%%%%%%%%%%%%%%%%%%%%%%
\section{Estimation:}
We are considering two main specifications for estimating teacher effect heterogeneity. Both of these methods build off of the existing approaches for calculating Value added like teacher fixed effects or average (dosage-adjusted) residuals and we'll compare our approaches to those and to the linear approximation used in \citet{lockwood2007sensitivity}.


%%%%%%%%%%%%%%%%%%%%%%%%%%%%%%%%%%%%%%%%%%%%%%%%%%%%%%%%
%%%%%%%%%%%%%%%%%%%%%%%%%%%%%%%%%%%%%%%%%%%%%%%%%%%%%%%%
\subsection{Weighted OLS Residuals and Empirical Bayes}
Our simplest way to estimate VAM heterogeneity is to re-weight the student residuals that would be used in estimating traditional (mean) VAM by student achievement. First we can run a regression of student post tests on student pretests and controls and capture the OLS residuals. Traditionally, the VAM would be the unweighted mean of these residuals\footnote{we would run a regression of the residuals on a ``dose matrix'' to decompose the effects across teachers.}. Instead  we could weight the student residuals here according to achievement and our welfare measure with weighted OLS and report a ``Welfare Added'' statistic. The result of this method is a scalar determined by student residuals, student achievement, and welfare weights. We could repeat this process with different weights to characterize value added over policies emphasizing different parts of the achievement distribution.

A less parametric option is binning students based on their prior year achievement (e.g., quartiles). We can then take the average difference between observed and predicted improvement in each bin to estimate for the current teacher's value added for each bin in the achievement distribution.

The biggest issue to overcome will be parsing out signal from noise especially given that cell sizes will tend to be small. Either of these approaches could be extended by adding empirical bayes shrinkage to the estimates. This would give us more practical point estimates in small sample sizes. 

An important theoretical hurdle for either  method is in defining what ``achievement'' is. For example is the ``lowest achieving'' student the student who has the lowest pre-period test score, the lowest (\textit{ex ante}) predicted post-period score, the lowest (\textit{ex post}) post period score,  or something else entirely? 


%%%%%%%%%%%%%%%%%%%%%%%%%%%%%%%%%%%%%%%%%%%%%%%%%%%%%%%%
%%%%%%%%%%%%%%%%%%%%%%%%%%%%%%%%%%%%%%%%%%%%%%%%%%%%%%%%
\subsection{Quantile Regression}
Another way we could estimate the heterogeneity of teacher value added is with a quantile regression. The most intuitive way to do this may be with a ``fixed effects approach,'' rather than the average residual approach. Specifically, we would estimate the quantile regression of student post-period scores on student controls, previous period scores, and teacher fixed effects.\footnote{Again this becomes complicated if students can possibly have multiple teachers who affect their same test scores.} Although there may be an residual-based analogue where we estimate the average residuals from the quantile regression estimates. In any case the idea would be to show how the estimate of a teacher's average value added change with the percentile at which we center the loss function. 

An advantage of using the quantile regression is that it simplifies the weighting problem introduced in the previous subsection. The quantile regression's ``check'' loss function will naturally weight observations by the percentile of their residual. This is a simple and intuitive weighting function. The disadvantage is that residuals may not be exactly what we want to use to characterize the achievement distribution---especially if we decide to include student characteristics as controls in addition to test scores from previous periods.

A potential disadvantage of the quantile approach is whether it works in high dimensional settings (i.e., with lots of fixed effects). Another potential issue comes from the difficulty to interpret quantile coefficients: Including covariates makes it so that we are no longer comparing people in the absolute quantiles, but rather within their quantile conditional on the X. For example, after including race indicators, the quantile results wouldn't correspond to the association at the (unconditional) first percentile, but rather a weighted average of the association for black students in the first percentile among black students and white students at the first percentile among white students.\footnote{It may be worth considering whether this could be engineered into a feature rather than a bug. For example by only including lagged scores as our X's, the results might be able to tell us the quantile effect for the worst achievers.}




%%%%%%%%%%%%%%%%%%%%%%%%%%%%%%%%%%%%%%%%%%%%%%%%%%%%%%%%
%%%%%%%%%%%%%%% Welfare and Simulations. %%%%%%%%%%%%%%%
%%%%%%%%%%%%%%%%%%%%%%%%%%%%%%%%%%%%%%%%%%%%%%%%%%%%%%%%
\section{Analysis and Simulations:}


%%%%%%%%%%%%%%%%%%%%%%%%%%%%%%%%%%%%%%%%%%%%%%%%%%%%%%%%
%%%%%%%%%%%%%%%%%%%%%%%%%%%%%%%%%%%%%%%%%%%%%%%%%%%%%%%%
\subsection{Econometric Simulations:}

The purpose of our first set of simulations will be to show which of the proposed estimation strategies best recover the true teacher VA across the achievement distribution in a variety of settings. We will measure the relative merits of each approach under a variety of data-generating processes, and suggest which perform best in settings that most resemble reality.

To accomplish this, we will consider data generating processes which vary classroom size, degree of heterogeneity in terms of prior achievement, the amount of noise in current-year test scores, correlation structure for ability and errors within classes, and teacher VA across the achievement distribution. This will allow us to compare the estimators under each different scenario. For example, we will be able to characterize how many classroom-year observations we would need per teacher in order to reliably estimate her VA at different points in the achievement distribution or how coarsely or finely each estimator can measure VA heterogeneity given intra-class correlations. 

These simulations will also show evidence about whether estimating heterogeneous VA would be practically useful in evaluating teacher effectiveness (e.g., if it takes 30 classroom/year observations with classes of average size to have desirable properties for these estimators, they will not be valuable tools for educational policy). They will also tell us under which situations certain approaches of estimation will dominate others.


%%%%%%%%%%%%%%%%%%%%%%%%%%%%%%%%%%%%%%%%%%%%%%%%%%%%%%%%
%%%%%%%%%%%%%%%%%%%%%%%%%%%%%%%%%%%%%%%%%%%%%%%%%%%%%%%%
\subsection{Theoretical Simulations:}

The second set of simulations would tell us the implications of utilizing the information from our estimates in education policy. Whereas the first set of simulations will be preformed on synthetic data, these will be most interesting if they are performed using the estimates from real linked student teacher data.

Because one of the declared intents of value added measures is to compare teachers, the main body of simulations we would do here would be to see how these comparisons change when value added is calculated in regards to different welfare functions. Interesting welfare criteria to analyze could include

\begin{itemize}
    \item \textbf{Equally Weighted} - The teacher is evaluated based on total value added (or equivalently mean value added) with each student equally weighted.
    \item \textbf{NCLB} - improvements to students at the bottom of the distribution are given more weight than those at the top. 
    \item \textbf{Rawlsian} - The teacher is evaluated based on total value added for the bottom student (or bottom XX\% of students) from ex-post test score distribution. 
    \item \textbf{Cutoff} - The teacher is evaluated based on total value added for only students below an ex-post test score cutoff.
    \item \textbf{Budget Efficiency} - If given some measure of the differential cost required to raise test scores for students across the achievement distribution, we can build a measure where teachers are evaluated based on the ``efficiency" of their effort allocation.
\end{itemize}

In addition to the above, it may be interesting to investigate teacher reoptimization under each separate welfare criterion to evaluate the distributional consequences of policies focused on the given criterion. Under the assumptions that the educational policies correctly evaluate teachers based the specified criterion and that teachers respond to the policies, these simulations would allow us to investigate what trade-offs policymakers face when deciding which welfare criteria they want to use in accountability policies.


%%%%%%%%%%%%%%%%%%%%%%%%%%%%%%%%%%%%%%%%%%%%%%%%%%%%%%%%
%%%%%%%%%%%%%%%%%%%%%%%%%%%%%%%%%%%%%%%%%%%%%%%%%%%%%%%%
\subsection{Substantive Simulations:}

Our last simulations will extend the previous analysis to examine whether a rearrangement of teachers and/or students could improve welfare as measured by the various criteria above simulated (a similar idea to what \citet{condie2014teacher} consider). Again, these simulations will be done with a particular eye to achieving the specific welfare goal specified, so we will make no normative statement about which welfare criteria is preferable. Not only will these simulations give us an idea of what gains could be achieved by purposeful sorting of students and teachers under the various welfare criteria, but they will allow us to speak to the welfare loss of the current sorting in each case. This last point would be particularly interesting in understanding how well the policy regime under which our data were collected actually achieved its goals. 




%%%%%%%%%%%%%%%%%%%%%%%%%%%%%%%%%%%%%%%%%%%%%%%%%%%%%%%%
%%%%%%%%%%%%%%%%%%%%%%% Data etc %%%%%%%%%%%%%%%%%%%%%%%
%%%%%%%%%%%%%%%%%%%%%%%%%%%%%%%%%%%%%%%%%%%%%%%%%%%%%%%%
\section{Data and Empirics:}

To accomplish this project appropriately, we will need to have student-teacher linked data over multiple years. Some possibilities we are considering are data from North Carolina, Los Angeles Unified School District, and San Diego Unified School District. These are all administrative data that would require application and approval, but these entities frequently work with researchers investigating questions of interest to the administration.

Our other data need is data with which we could try to estimate and/or understand the average relative effort to raise the test scores of students at different points in the achievement distribution. This could come, at least partially from qualitative interviews with teachers. It is also possible that we could estimate the revealed effort required using data from classrooms where a policy was in effect that rewarded total value added. Either of these would allow us to have a rough estimate of the effort required so that we could begin to speak to efficiency of teacher effort.



%%%%%%%%%%%%%%%%%%%%%%%%%%%%%%%%%%%%%%%%%%%%%%%%%%%%%%%%
%%%%%%%%%%%%%%%%%%%%%% References %%%%%%%%%%%%%%%%%%%%%%
%%%%%%%%%%%%%%%%%%%%%%%%%%%%%%%%%%%%%%%%%%%%%%%%%%%%%%%%
\bibliography{citations}
    

\end{document}


% Chris comment: can you horse race va on different outcomes to see what matters at different points of the distribution.