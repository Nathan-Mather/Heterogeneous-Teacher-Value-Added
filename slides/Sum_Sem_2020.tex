%%%%%%%%%%%%%%%%%%%%%%%%%%%%%%%%%%%%%%%%%
% Beamer Presentation
% LaTeX Template
% Version 1.0 (10/11/12)
%
% This template has been downloaded from:
% http://www.LaTeXTemplates.com
%
% License:
% CC BY-NC-SA 3.0 (http://creativecommons.org/licenses/by-nc-sa/3.0/)
%
%%%%%%%%%%%%%%%%%%%%%%%%%%%%%%%%%%%%%%%%%

%----------------------------------------------------------------------------------------
%	PACKAGES AND THEMES
%----------------------------------------------------------------------------------------
\documentclass[t,aspectratio=169,11pt]{beamer}
%\documentclass[t,aspectratio=169,11pt,draft]{beamer}

\mode<presentation> {

\input{LatexColors.incl.tex}

\usetheme{Boadilla}

% Footer and page numbers

% First remove the navigation symbols from the bottom of all slides 
\setbeamertemplate{navigation symbols}{} 
% Make a new prettier page number
\addtobeamertemplate{navigation symbols}{}{%
    \usebeamerfont{footline}%
    \hspace{5em}%
    {\color{black!50}{\small\insertframenumber/\inserttotalframenumber}}%
}
% Remove the colored line at the bottom
\setbeamertemplate{footline}{}


% Customize items
\setbeamertemplate{itemize items}{\color{ptlightblue}{\rule[0.05em]{0.6em}{0.6em}}}
\setbeamertemplate{itemize subitem}{\color{ptlightblue}{\raisebox{0.12em}{$\blacktriangleright$}}}
\setbeamercolor{enumerate item}{fg=black}
\setbeamertemplate{enumerate items}[default]
\setbeamercolor{enumerate subitem}{fg=black}
\setbeamertemplate{enumerate subitems}[default]


% Table of Contents (if we have one)
\setbeamertemplate{section in toc}{\inserttocsectionnumber.~\inserttocsection}
\setbeamercolor{section in toc}{bg=white,fg=black}
\setbeamertemplate{subsection in toc}{%
  \hspace{1.2em}{{\color{ptlightblue}{\rule[0.05em]{0.6em}{0.6em}}}~\inserttocsubsection\par}
}
\setbeamercolor{subsection in toc}{bg=white,fg=black}


% Customize beamer colors
\setbeamercolor{titlelike}{fg=white,bg=ptlightblue}
\setbeamercolor{button}{bg=ptlightcyan,fg=white}
}

% Nicer looking Itemize and enumerate environments
\newenvironment{wideitemize}{\itemize\addtolength{\itemsep}{14pt}}{\enditemize}
\newenvironment{wideenumerate}{\enumerate\addtolength{\itemsep}{14pt}}{\endenumerate}

% Import packages
\usepackage[default]{lato} % More modern looking font than Helvetica
\usepackage{geometry,ulem,caption,color,setspace,dsfont,physics,commath,amsmath,amssymb}
\usepackage[comma]{natbib}
\usepackage[short]{optidef}
\usepackage{hhline}
\usepackage{array}
\usepackage{booktabs} % Allows the use of \toprule, \midrule and \bottomrule in tables
\usepackage{adjustbox}
\usepackage{graphicx,xcolor,bbm,xcomment}
\usepackage{appendixnumberbeamer}
\usepackage{textcomp}
\usepackage{colortbl}
\usepackage{subcaption} 
\usepackage{tikz}
\usetikzlibrary{calc,patterns,positioning}
\usepackage{pdflscape}
\usepackage{environ}

\bibliographystyle{ecta}



% Make Section Header Frames
\AtBeginSection[]{
  \begin{frame}
  \vfill
  \centering
  \begin{beamercolorbox}[sep=8pt,center,shadow=true,rounded=true]{title}
    \usebeamerfont{title}\insertsectionhead\par%
  \end{beamercolorbox}
  \vfill
  \end{frame}
}

\AtBeginSubsection[]{
  \begin{frame}
  \vfill
  \centering
  \begin{beamercolorbox}[sep=8pt,center,shadow=true,rounded=true]{title}
    \usebeamerfont{title}\insertsectionhead: \\ \insertsubsectionhead\par%
  \end{beamercolorbox}
  \vfill
  \end{frame}
}



\tikzset{
    invisible/.style={opacity=0},
    visible on/.style={alt={#1{}{invisible}}},
    alt/.code args={<#1>#2#3}{%
      \alt<#1>{\pgfkeysalso{#2}}{\pgfkeysalso{#3}} % \pgfkeysalso doesn't change the path
    },
  }



%----------------------------------------------------------------------------------------
%	TITLE PAGE
%----------------------------------------------------------------------------------------

\title{From Value Added to Welfare Added} 

\author{Tanner Eastmond \hfill \and Nathan Mather   \and \hfill Michael D Ricks   \\   \textit{teastmond@ucsd.edu}\hfill \textit{njmather{\fontfamily{qag}\selectfont @}umich.edu} \hfill
\textit{ricksmi{\fontfamily{qag}\selectfont @}umich.edu} } 
\institute{}
\date{\today} 

\begin{document}

\begin{frame}
\titlepage 
\end{frame}

%------------------------------------------------
% Why do we care 
\begin{frame}{Use of Value Added Measures Is Proliferating}
\begin{wideitemize}
    \item 
    
    \item 


\end{wideitemize}

\end{frame}
%------------------------------------------------

%------------------------------------------------
% What do we know and not know
\begin{frame}{We Don't Know To What Extend Heterogeneity Matters}

\begin{wideitemize}
    \item We know teachers respond to incentives
    \begin{itemize}
        \item {\color{gray}{\citep{jacob2003rotten,NCLB,pope2017multidimensional}}}
    \end{itemize}
    \item We know VAM capture something real
    \begin{itemize}
        \item {\color{gray}{\citep{chetty2014measuring2,pope2017multidimensional}}}
    \end{itemize}
    \item We know there is heterogeneity VAM do not catch
    \begin{itemize}
        \item {\color{gray}{\citep{ehrenberg1995teachers,dee2005teacher}}}
    \end{itemize}
    % This means we aren't necessarily incentivizing everything we want to be
    
    \vfill
    \item<2-> \textbf<2>{How much does teacher value added vary across the achievement distribution?} %We are intersted in this within and across teachers right?
    \vspace{-14pt}
    \item<3-> \textbf<3>{What does this mean for welfare and policy?}
    \vfill
\end{wideitemize}
\end{frame}

%------------------------------------------------
% How do we answer our question?
\begin{frame}{We Estimate VA Heterogeneity Along the Achievement Distribution}


\begin{wideenumerate}
     \item Frame VAM (and other educational statistics) in a welfare-relevant framework

    \item \textbf<2>{Develop new tools to estimate value added beyond mean effects}
    % Tied to (derived from the welfare approaches)
    % Econometric simulations to assess performance
    \item Employ methods in real data
    \begin{itemize}
        \item Estimate VAM across achievement distribution
        \item Investigate heterogeneity in long-term outcomes 
        \item Analyze rank inversions and implications for welfare
        \item Can we learn something about effort or teachers PPFs?
        \item Policy simulations to consider implications of policies
    \end{itemize}  

\end{wideenumerate}
\end{frame}

%------------------------------------------------
% What is the answer
\begin{frame}{Preliminary Results Look Promising}


\begin{wideitemize}
\item Today we'll report on some simple simulation exercises:
\begin{itemize}
    \item<2-> 
    \item<3-> 
\end{itemize}
\item<4-> 
\end{wideitemize}
\end{frame}

%-----------------------------------------------
\begin{frame}[c]{Today's Talk}

\begin{wideenumerate}
    \item Big picture, does this look like it is worth pursuing?
    \item Next steps (or steps we haven't thought about)?
    \item What first order concerns should we be aware of and thinking about?

\end{wideenumerate}
\end{frame}


%------------------------------------------------
\section{Welfare and Value Added}
%------------------------------------------------
%------------------------------------------------
\begin{frame}{Modeling Value Added As Social Welfare}

\begin{wideitemize}
    \item We can think of the following social welfare function (discrete for now)
    \begin{itemize}
        \item $\omega(x_i,y_i)$ weights: likely based on \textit{ex ante} expected performance
        % Talk about the concavity here?
        \item $v(x_i,y_i)$ value: think achievement, gains, etc.
    \end{itemize}
    \[
    W  = \sum_i \omega(x_i,y_i) v(x_i,y_i) 
    \] 
    
    \item Value added scores take the average gains for students among each teacher
    \begin{itemize}
        \item If $x_i$ is \textit{ex ante} expected performance, $v(\cdot) = y_i - x_i$
    \end{itemize}
    \[
    W_{VA}  = \sum_i \frac{(y_i-x_i)}{N_{j(i)}} \hspace{3em}
    \]
    
    \item This implies $\omega(x_i,y_i)=\frac{1}{N_j}$, which is (almost) utilitarian
    % this also implies that the value we care about is growth above/compared to similar students. What I mean is theoretically the same measures could be achieved by changing V or W, but changing W seams easier.  
    % Not what we hear people talking about 
    % For reference a VA version of NCLB has a weight  .... 
    
    
    % Note VA scores are mean zero (almost--because you are averaging over classes of differnt sizes) , so total welfare is uninformative but rank is informative.
\end{wideitemize}


\end{frame}
%------------------------------------------------
\begin{frame}{Estimating Value Added}

\begin{wideitemize}
    \item In theory test scores are a function of testing ability, value added, and noise
    
    \begin{align*}
    y_{ijt}  &= a_{it} + \epsilon_{ijt} \\
    y_{ijt}  &= a_{it-1} + VA_j(a_{it-1}) + \epsilon_{ijt}
    \end{align*}
 
    \item We don't observe testing ability. Traditional assumptions
    \begin{itemize}
        \item Predict testing ability with characteristics (including lagged scores): $X_{it}$
        \item Assume homogeneity within and across teachers $VA_j=\gamma d_j$ with no $ij$ error
    \end{itemize}
        
    \begin{align*}
    y_{ijt}  &= \beta X_{it} +\Gamma D_{it} +\mu_{jt} + \eta_{it}
    \end{align*}
    
    \item {\color{gray}{This might be the other place to talk about the $\mu_{jt}$ term picking up ij variation.}}
       
    
\end{wideitemize}
\end{frame}

%------------------------------------------------
\begin{frame}{[Plot corr($VA$,$\hat{VA}$) against var($\eta_{it}$)]} % or \mu_{jt}?

\centering
\scalebox{.8}{\parbox{\linewidth}{%   
    \begin{align*}
    estimating equation
    \end{align*}
}}
\includegraphics[width=.85\linewidth]{Slides/a.pdf}

\end{frame}
%------------------------------------------------
\begin{frame}{[Plot pct $\hat{VA}$ outside of 95\% CI of lowest) against var($\eta_{it}$) or $\mu_{jt}$?]} 
\centering
\scalebox{.8}{\parbox{\linewidth}{%   
    \begin{align*}
    estimating equation
    \end{align*}
}}
\includegraphics[width=.85\linewidth]{Slides/a.pdf}

\end{frame}

%------------------------------------------------
%------------------------------------------------
\section{Measuring Heterogeneity in Teacher Effects}
%------------------------------------------------
%------------------------------------------------


\begin{frame}{Motivation from the Social Welfare function}

\begin{wideitemize}
    \item Imagine the SP only cares about students exactly 1.5$\sigma$ below average
    
    \item<2-> Two problems with binning students by \textit{ex ante} expected score and estimating VA
    \begin{enumerate}
        \item<3-> We don't have infinite data, so bins would be too small
        \item<4-> Even if we did, we don't know the \textit{ex ante} expected score
    \end{enumerate}

    \item<5-> $\hat{y}_{it} = \hat{\beta}_1X_i + \sum \hat{\beta}_jy_{ij-1}$ consistent for expected score
    \begin{itemize}
        \item But prediction error means that binning no longer works %(do we need to formalize this? That the ammount of students who SP cares about should be decreasing from that point out)
    \end{itemize}

    \item<6-> We propose three methods that (not only deal with but) capitalize on this
    \begin{enumerate}
        \item WLS regression with weights defined by the distance from the target (e.g., -1.5$\sigma$)
        \item Quantile regression evaluated at the relevant percentile
        \item Kernel regression of effect of $D_j$ over achievement distribution
    \end{enumerate}

\end{wideitemize}



\end{frame}
%----------------------------------

%% It might be easier to not call this method one because it adds in a step of complication with having to do the Frisch-Waugh-Lovell projection.
\begin{frame}{Method One: Weighted Least Squares}

\begin{wideitemize}
    \item WLS is consistent in models ``fully saturated'' with binary  variables  \citet{solon2015we} 
    % i.e.,  it is consistent for average partial effects in the population represented by the reweighted sample.
    \begin{enumerate}
        \item<2-> Predict $y$ with $X$ and lagged $y$ and assign weights $W_p$
        % weights target the point in the achievemnet distribution of interest
        \item<3-> Use Frisch-Waugh-Lovell to partial out these predictors from $D_j$
        \item<4-> Estimate effects of teachers, $D_j$, weighting by predicted achievement
    \end{enumerate}
    
    \begin{align*}
    Y &= \beta X + \Gamma D + \eta \\
    \\
    \alt<3-4>{(\hat{\beta}_{OLS},\hat{\Gamma}_{OLS}) &= \left ((\beta,\Gamma)'(\beta,\Gamma) \right ) ^{-1} (\beta,\Gamma)' Y \\
    \hat{\Gamma}_{OLS} &= \hat{\Gamma}_{FWL} =  (\tilde{\Gamma}'\tilde{\Gamma}  ) ^{-1} \tilde{\Gamma}' Y \\}%
    {\\
    \\}
    \\
    \alt<4>{\hat{\Gamma}_{p} &= \left (\tilde{\Gamma}'W_p\tilde{\Gamma} \right ) ^{-1} \tilde{\Gamma}'W_p Y}{} \\
    \end{align*}
    

\end{wideitemize}



\end{frame}

%----------------------------------


\begin{frame}{Method Two: Quantile Regression}

\begin{wideitemize}

    \item Quantile regressions recover the ... \citep{}
    \begin{enumerate}
        %\item Predict $y$ with $X$ and lagged $y$ and assign $\hat{y}$
        \item Estimate the effect of teachers $D_j$ at the $p$th quantile (conditional on X)
        % There is a tension here about whether we want the Xs to be evaluated at the given quantile or not...
        %\item \color{gray}{or Use Frisch-Waugh-Lovell to partial out mean predictors then estimate effects of teachers on students at the (unconditional) $p$th quantile]?}
        
    \end{enumerate}
    \item {\color{gray}{Is it really that simple?}}
    \item Would the SP care about conditioning on X at the $p$th quantile or on average?
    \item How to estimate many fixed effects in quantile models? 
    \item Conditional quantile on $X$ needed, but not on $D$? or also on $D$?

\end{wideitemize}
\end{frame}

%----------------------------------


\begin{frame}{Method Three: Kernel Regression}

\begin{wideitemize}

    \item Assuming continuity, the best estimate for the value of $y$ at a point $x$,  is using a kernel
    \item Kernel weighting has a theoretical analogue: 
    \begin{itemize}
        \item The welfare for a SP who only cares about students at $x$ (but doesn't observe types)
    \end{itemize}
 
    
    % I'm not sure technically how to implement this.. Is it E[Y|\hat{Y}] for each D?
    \item ... \citep{}
    \begin{itemize}
        \item Predict $y$ with $X$ and lagged $y$ and assign $\hat{y}$
        \item For each $j$ estimate $\mathbb{E}[y|\hat{y}]$ nonparametrically
    \end{itemize}

    \item Not re-estimating the model under different weights (or objective functions)
    \item More data hungry \color{gray}{(when to talk about heterogeneity vs changes over time?)
    \item Bandwidth selection...? (Also a place to talk about measuring scores in pctle vs $\sigma$?)}

\end{wideitemize}
\end{frame}
%----------------------------------

\begin{frame}{[Plot estimate against the truth for one or two sets of weights ]} 
\centering
\scalebox{.8}{\parbox{\linewidth}{%   
    \begin{align*}
    estimating equation
    \end{align*}
}}
\includegraphics[width=.85\linewidth]{Slides/a.pdf}

\end{frame}
%----------------------------------

\begin{frame}{[Plot corr(VA(W),$\hat{VA}(W)$) over $W$]} 
\centering
\scalebox{.8}{\parbox{\linewidth}{%   
    \begin{align*}
    estimating equation
    \end{align*}
}}
\includegraphics[width=.85\linewidth]{Slides/a.pdf}

\end{frame}
%----------------------------------


\begin{frame}{[Caterpillar Plot (for what weight?)]}

\centering
\scalebox{.8}{\parbox{\linewidth}{%   
    \begin{align*}
    estimating equation
    \end{align*}
}}
\includegraphics[width=.85\linewidth]{Slides/a.pdf}

\end{frame}
%----------------------------------


\begin{frame}{[Plot number of rank inversions (or square loss) over $W$}

\centering
\scalebox{.8}{\parbox{\linewidth}{%   
    \begin{align*}
    estimating equation
    \end{align*}
}}
\includegraphics[width=.85\linewidth]{Slides/a.pdf}

\end{frame}
%----------------------------------


\begin{frame}{[Some figures (and or tables) exploring how $\mathbb{E}[N_j]$ and var($\eta$) ]}

\centering
%\includegraphics[width=.9\linewidth]{Slides/}


\end{frame}
%------------------------------------------------
%------------------------------------------------
\section{Feedback and Next Steps}

%------------------------------------------------
\begin{frame}{Next Steps}

\begin{wideitemize}
    \item 
\end{wideitemize}

\end{frame}


%------------------------------------------------
\begin{frame}{Questions and Feedback}

\begin{wideitemize}
    \item 
    \item<2-> 
    \begin{itemize}
        \item 
        \item 
    \end{itemize}
    \item<3-> 
    \vspace{12pt}
    \item<4-> \textbf{What are your questions, suggestions, and thoughts?} 
    

\end{wideitemize}

\end{frame}



%------------------------------------------------


\begin{frame}[c]
\centering
\Huge{\centerline{To Be Continued...}}
\normalsize njmather{\fontfamily{qag}\selectfont @}umich.edu
ricksmi{\fontfamily{qag}\selectfont @}umich.edu
\end{frame}

%------------------------------------------------


\begin{frame}
\frametitle{References}
\tiny
\bibliography{citations}
\end{frame}


%----------------------------------------------------------------------------------------
%----------------------------------------------------------------------------------------

\appendix
\setbeamertemplate{navigation symbols}{}


%----------------------------------------------------------------------------------------
%----------------------------------------------------------------------------------------


\begin{frame}{}

\end{frame}






\end{document}



