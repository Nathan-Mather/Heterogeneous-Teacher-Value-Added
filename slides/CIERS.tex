%%%%%%%%%%%%%%%%%%%%%%%%%%%%%%%%%%%%%%%%%
% Beamer Presentation
% LaTeX Template
% Version 1.0 (10/11/12)
%
% This template has been downloaded from:
% http://www.LaTeXTemplates.com
%
% License:
% CC BY-NC-SA 3.0 (http://creativecommons.org/licenses/by-nc-sa/3.0/)
%
%%%%%%%%%%%%%%%%%%%%%%%%%%%%%%%%%%%%%%%%%

%----------------------------------------------------------------------------------------
%	PACKAGES AND THEMES
%----------------------------------------------------------------------------------------
\documentclass[t,aspectratio=169,11pt]{beamer}
%\documentclass[t,aspectratio=169,11pt,draft]{beamer}

\mode<presentation> {

\input{LatexColors.incl.tex}

\usetheme{Boadilla}

% Footer and page numbers

% First remove the navigation symbols from the bottom of all slides 
\setbeamertemplate{navigation symbols}{} 

% Remove the colored line at the bottom
\setbeamertemplate{footline}{}


% Customize items
\setbeamertemplate{itemize items}{\color{ptlightblue}{\rule[0.05em]{0.6em}{0.6em}}}
\setbeamertemplate{itemize subitem}{\color{ptlightblue}{\raisebox{0.12em}{$\blacktriangleright$}}}
\setbeamercolor{enumerate item}{fg=black}
\setbeamertemplate{enumerate items}[default]
\setbeamercolor{enumerate subitem}{fg=black}
\setbeamertemplate{enumerate subitems}[default]


% Table of Contents (if we have one)
\setbeamertemplate{section in toc}{\inserttocsectionnumber.~\inserttocsection}
\setbeamercolor{section in toc}{bg=white,fg=black}
\setbeamertemplate{subsection in toc}{%
  \hspace{1.2em}{{\color{ptlightblue}{\rule[0.05em]{0.6em}{0.6em}}}~\inserttocsubsection\par}
}
\setbeamercolor{subsection in toc}{bg=white,fg=black}


% Customize beamer colors
\setbeamercolor{titlelike}{fg=white,bg=ptlightblue}
\setbeamercolor{button}{bg=ptlightcyan,fg=white}
}


% Import packages
\usepackage[default]{lato} % More modern looking font than Helvetica
\usepackage{geometry,ulem,caption,color,setspace,dsfont,physics,commath,amsmath,amssymb}
\usepackage[comma]{natbib}
\usepackage[short]{optidef}
\usepackage{hhline}
\usepackage{array}
\usepackage{booktabs} % Allows the use of \toprule, \midrule and \bottomrule in tables
\usepackage{adjustbox}
\usepackage{graphicx,xcolor,bbm,xcomment}
\usepackage{appendixnumberbeamer}
\usepackage{textcomp}
\usepackage{colortbl}
\usepackage{subcaption} 
\usepackage{tikz}
\usetikzlibrary{calc,patterns,positioning}
\usepackage{pdflscape}
\usepackage{environ}

\bibliographystyle{ecta}



% Make Section Header Frames
\AtBeginSection[]{
  \setbeamertemplate{navigation symbols}{}
  \begin{frame}[noframenumbering]
  \vfill
  \centering
  \begin{beamercolorbox}[sep=8pt,center,shadow=true,rounded=true]{title}
    \usebeamerfont{title}\insertsectionhead\par%
  \end{beamercolorbox}
  \vfill
  \end{frame}

% Make a new prettier page number
\addtobeamertemplate{navigation symbols}{}{%
    \usebeamerfont{footline}%
    \hspace{5em}%
    {\color{black!50}{\small\insertframenumber/\inserttotalframenumber}}%
}
  
}


\tikzset{
    invisible/.style={opacity=0},
    visible on/.style={alt={#1{}{invisible}}},
    alt/.code args={<#1>#2#3}{%
      \alt<#1>{\pgfkeysalso{#2}}{\pgfkeysalso{#3}} % \pgfkeysalso doesn't change the path
    },
  }

    
    %--------------------------------------------------------------
    % Fix item spacing
    %----------------------------------------------------------------
    
         % redefine itemsep to autofill space on beamer slides
    % % from https://tex.stackexchange.com/questions/369504#369597
    \makeatletter
    \renewcommand{\itemize}[1][]{%
    	\beamer@ifempty{#1}{}{\def\beamer@defaultospec{#1}}%
    	\ifnum \@itemdepth >2\relax\@toodeep\else
    	\advance\@itemdepth\@ne
    	\beamer@computepref\@itemdepth% sets \beameritemnestingprefix
    	\usebeamerfont{itemize/enumerate \beameritemnestingprefix body}%
    	\usebeamercolor[fg]{itemize/enumerate \beameritemnestingprefix body}%
    	\usebeamertemplate{itemize/enumerate \beameritemnestingprefix body begin}%
    	\list
    	{\usebeamertemplate{itemize \beameritemnestingprefix item}}
    	{\def\makelabel##1{%
    			{%
    				\hss\llap{{%
    						\usebeamerfont*{itemize \beameritemnestingprefix item}%
    						\usebeamercolor[fg]{itemize \beameritemnestingprefix item}##1}}%
    			}%
    		}%
    	}
    	\fi%
    	\setlength\itemsep{\fill}
    	\ifnum \@itemdepth >1
    	\vfill
    	\fi%  
    	\beamer@cramped%
    	\raggedright%
    	\beamer@firstlineitemizeunskip%
    }
    \def\enditemize{\ifhmode\unskip\fi\endlist%
    	\usebeamertemplate{itemize/enumerate \beameritemnestingprefix body end}
    	\ifnum \@itemdepth >1
    	\vfil
    	\fi%  
    }
    \makeatother
    
    \makeatletter
    \def\enumerate{%
    	\ifnum\@enumdepth>2\relax\@toodeep
    	\else%
    	\advance\@enumdepth\@ne%
    	\edef\@enumctr{enum\romannumeral\the\@enumdepth}%
    	\advance\@itemdepth\@ne%
    	\fi%
    	\beamer@computepref\@enumdepth% sets \beameritemnestingprefix
    	\edef\beamer@enumtempl{enumerate \beameritemnestingprefix item}%
    	\@ifnextchar[{\beamer@@enum@}{\beamer@enum@}}
    \def\beamer@@enum@[{\@ifnextchar<{\beamer@enumdefault[}{\beamer@@@enum@[}}
    \def\beamer@enumdefault[#1]{\def\beamer@defaultospec{#1}%
    	\@ifnextchar[{\beamer@@@enum@}{\beamer@enum@}}
    \def\beamer@@@enum@[#1]{% partly copied from enumerate.sty
    	\@enLab{}\let\@enThe\@enQmark
    	\@enloop#1\@enum@
    	\ifx\@enThe\@enQmark\@warning{The counter will not be printed.%
    		^^J\space\@spaces\@spaces\@spaces The label is: \the\@enLab}\fi
    	\def\insertenumlabel{\the\@enLab}
    	\def\beamer@enumtempl{enumerate mini template}%
    	\expandafter\let\csname the\@enumctr\endcsname\@enThe
    	\csname c@\@enumctr\endcsname7
    	\expandafter\settowidth
    	\csname leftmargin\romannumeral\@enumdepth\endcsname
    	{\the\@enLab\hspace{\labelsep}}%
    	\beamer@enum@}
    \def\beamer@enum@{%
    	\beamer@computepref\@itemdepth% sets \beameritemnestingprefix
    	\usebeamerfont{itemize/enumerate \beameritemnestingprefix body}%
    	\usebeamercolor[fg]{itemize/enumerate \beameritemnestingprefix body}%
    	\usebeamertemplate{itemize/enumerate \beameritemnestingprefix body begin}%
    	\expandafter
    	\list
    	{\usebeamertemplate{\beamer@enumtempl}}
    	{\usecounter\@enumctr%
    		\def\makelabel##1{{\hss\llap{{%
    						\usebeamerfont*{enumerate \beameritemnestingprefix item}%
    						\usebeamercolor[fg]{enumerate \beameritemnestingprefix item}##1}}}}}%
    	\setlength\itemsep{\fill}
    	\ifnum \@itemdepth >1
    	\vfill
    	\fi%  
    	\beamer@cramped%
    	\raggedright%
    	\beamer@firstlineitemizeunskip%
    }
    \def\endenumerate{\ifhmode\unskip\fi\endlist%
    	\usebeamertemplate{itemize/enumerate \beameritemnestingprefix body end}
    	\ifnum \@itemdepth >1
    	\vfil
    	\fi%  
    }
    \makeatother
    


%----------------------------------------------------------------------------------------
%	TITLE PAGE
%----------------------------------------------------------------------------------------


\title{From Value Added to Welfare Added} 

\author{Tanner Eastmond \hfill \and Nathan Mather   \and \hfill Michael D Ricks   \\   \textit{teastmond@ucsd.edu}\hfill \textit{njmather{\fontfamily{qag}\selectfont @}umich.edu} \hfill
\textit{ricksmi{\fontfamily{qag}\selectfont @}umich.edu} } 
\institute{}
\date{\today} 

\begin{document}

\begin{frame}[noframenumbering]
\titlepage 
\end{frame}

% Make a new prettier page number
\addtobeamertemplate{navigation symbols}{}{%
    \usebeamerfont{footline}%
    \hspace{5em}%
    {\color{black!50}{\small\insertframenumber/\inserttotalframenumber}}%
}

%------------------------------------------------
% What is Value added 
%------------------------------------------------
\begin{frame}{Value Added Measures: Intent and Reality}
\begin{itemize}
    \item Goal of Value added measures (VAM) are to rank teachers' ``effectiveness''
        \begin{itemize}
            \item Measure post test achievement controlling for a pretest score
        \end{itemize}
    \item We know VAM do capture something real
    \begin{itemize}
        \item {\color{gray}{\citep{chetty2014measuring2,pope2017multidimensional}}}
    \end{itemize}
    \item We know there is heterogeneity in test score effects VAM do not catch
    \begin{itemize}
        \item {\color{gray}{\citep{ehrenberg1995teachers,dee2005teacher}}}
    \end{itemize}
    % \item We know that if we use VAM for incentives has potential and limitations
    %\begin{itemize}
    %    \item {\color{gray}{\citep{jacob2003rotten,neal2010left,pope2019effect}}}
    %\end{itemize}
\end{itemize}
    \vspace{48pt}
\vfill
\end{frame}

%------------------------------------------------
% Why do we care 
%------------------------------------------------
\begin{frame}{Value Added Measures: Policy Disconnect }
\begin{itemize}


    \item Theoretical disconnect between education policy goals and use of VAM 
    \begin{itemize}
        \item Policies explicitly target gains in some subpopulations
        \item VAM are mean-oriented statistics: a teachers average impact on students' scores
        \item Resulting VAM rankings may carry different \textbf{implicit normative} ``welfare weights''
    \end{itemize}
    
    \item  We propose a set of more flexible heterogeneous VAM 
    \begin{itemize}
        \item Assess teachers' heterogeneous impacts along the achievement distribution
        \item  Weight effects according to an \textbf{explicit normative} policy goal or welfare criterion
    \end{itemize}
\end{itemize}


\end{frame}

%------------------------------------------------
% How do we answer our question?
%------------------------------------------------

\begin{frame}{Research Goals}

\begin{enumerate}
     \item Build a welfare-relevant framework for VAM (and other educational statistics) 
     %\begin{itemize}
     %    \item Can we accommodate any policy or normative goal?
     %\end{itemize}
    % two to four two is method.
    \item Propose methods to estimate value added over the achievement distribution
        %\begin{itemize}
        %    \item Do certain teachers excel at teaching certain students?
        %\end{itemize}
    \item Explore feasibility and accuracy of methods in simulations
    %\begin{itemize}
    %    \item Are these methods feasible? Accurate? Informative? 
    %\end{itemize}
    \item Estimate VA heterogeneity and welfare implications in real student-teacher data
    %\begin{itemize}
    %    \item What implications do rank inversions have for welfare?
    %    \item What implications are there for possible policies?
    %\end{itemize}  
    % Policy relevant tool. When does heterogeneity matter?
\end{enumerate}
\end{frame}

%-----------------------------------------------
% Goal of toady's talk 
%-----------------------------------------------
\begin{frame}{Today's Talk}

\begin{enumerate}
    \item Are these research goals interesting?
    \item How can we make the methods and metrics most useful? 
    \item How convincing do the simulations suggest our methods are?
    \item What are the biggest concerns we should be thinking about?
\end{enumerate}
\end{frame}




%------------------------------------------------------------------------------------------
%------------------------------------------------------------------------------------------
% NEW SECTION: Setting up the theory 
%------------------------------------------------------------------------------------------
%------------------------------------------------------------------------------------------
\section{Measuring Welfare from Value Added}

%------------------------------------------------
% Estimating STANDARD Value Added 
%------------------------------------------------

\begin{frame}{Estimating Standard  Value Added}

\begin{itemize}
    \item In theory test scores are a function of testing ability, value added, and noise
    \begin{align*}
    y_{i,t}  &= a_{i,t} + \epsilon_{i,t} \\
    y_{i,t}  &= a_{i,t-1} + VA_{j(i)}(a_{i,t-1}) + \epsilon_{i,t}
    \end{align*}
 
    \item We don't observe testing ability. Traditional VAM assumptions:
    \begin{itemize}
        \item Predict testing ability with characteristics (including lagged scores): $X_{i,t}$
        \item Assume homogeneity within and across teachers $VA_{j}(\cdot)=\gamma_{j,t}$ with no $i,j$ error
    \end{itemize}
    \begin{align*}
    y_{i,t}  &= \beta X_{i,t} +\Gamma D_{i,t} + \eta_{i,t}
    \end{align*}
    
    \item Note $\gamma_{j,t}$ variation over time could pick up $i,j$ variation.
       
    
\end{itemize}
\end{frame}

%------------------------------------------------
% Obligatory model going over the logic 
%------------------------------------------------
\begin{frame}{Modeling Value Added As Social Welfare}

\begin{itemize}
    \item We can think of the following social welfare function
    \begin{itemize}
        \item $\omega(x_i)$ weights: based on \textit{ex ante} expected performance (pretest)
        % Talk about the concavity here?
        \item $v(x_i,y_i)$ value: think achievement, gains, etc.
    \end{itemize}
    \[
    W  = \sum_i \omega(x_i) v(x_i,y_i) 
    \] 
    
    \item Traditional VAM take the average gains for students among each teacher
    \begin{itemize}
        \item Expected performance is estimated as $\hat{y}_i$, then $\tilde{v}(\cdot) = y_i - \hat{y}_i$
    \end{itemize}
    \[
    \hat{W}_{j}  = \sum_{i\in\mathcal{C}_j} \frac{(y_i-\hat{y}_i)}{N_{j}} \hspace{3em}
    \]
    
    \item Implies a cardinal ``Value" equal to test score gains
    
    \item This implies $\omega(x_i)=\frac{1}{N_j}$, which is (almost) utilitarian
    % this also implies that the value we care about is growth above/compared to similar students. What I mean is theoretically the same measures could be achieved by changing V or W, but changing W seams easier.  
    % Not what we hear people talking about 
    % For reference a VA version of NCLB has a weight  .... 
    
    
    % Note VA scores are mean zero (almost--because you are averaging over classes of differnt sizes) , so total welfare is uninformative but rank is informative.
\end{itemize}


\end{frame}



\begin{frame}{``Welfare Added''}

\begin{itemize}
    \item Estimating any other Welfare Added requires an estimate of $\mathbb{E}[v_j(\cdot)|x_i]$
    \item Let $\hat{v}_j(x_i,y_i)$ estimate a teacher's heterogeneous value added, $\mathbb{E}[v_j(\cdot)|x_i]$
    
    \item Then the estimated ``Welfare Added'' by this teacher is  
    \[
    \hat{W}_j  = \sum_{i\in \mathcal{C}_j} \omega(x_i) \hat{v}_j(x_i,y_i) 
    \] 
    
    \item Consider the following welfare weight examples:
    % Should update this with whatever we actually end up using!!!!!!
    \begin{itemize}
        \item Utilitarian: All students are weighted equally $\omega(x_i) = 1$
        \item Rawlsian: We care more about students below some cutoff, $c$, $\omega(x_i) = \mathds{1}(x_i\leq c)$
        \item Pareto: We care about the lower achieving students more $ \omega(x_i)=  \frac{\alpha x_\mathrm{m}^\alpha}{x_i^{\alpha+1}}$ 
    \end{itemize}
    
\end{itemize}


\end{frame}

%-----------------------------------------------
% Motivation for Weighted VAM 
%-----------------------------------------------

\begin{frame}{Estimating Effect Heterogeneity}

\begin{itemize}
    \item A hypothetical ideal would be to estimate every $\hat{v}_j(x_i,y_i)$ separately
    \item This is impossible, but we propose three alternatives:
    \begin{enumerate}
        \item Bin students by pretest scores
        \item Nonparametrically estimate the impact of teachers over the achievement distribution
        \item Estimate quantile regressions over relevant percentiles
    \end{enumerate}

\end{itemize}

\end{frame}



%------------------------------------------------
% Method details 
%-------------------------------------------------


\begin{frame}{Using Bin Dummies}

\begin{itemize}
    \item Bin students into groups based on pretest
        \begin{itemize}
        \item Five equal bins across pretest quantiles (i.e. 0-20th, 21-40th, etc.)
        \item Let $Q_{i,t}$ indicate quantiles by teachers
        \begin{itemize}
          \item Note that not all bins necessarily have data 
        \end{itemize}
        \end{itemize}
     \begin{align*}
    y_{i,t}  &= \beta y_{i,t-1} + \sum_{n=1}^{5} \Gamma_n Q_{i,t} + \eta_{i,t}
    \end{align*}
    
    \item Get estimates for teachers on each bin 
    
    \item Weight those impacts according to our weights 
    
      \begin{align*}
    \hat{W}_j = \sum_{i\in \mathcal{C}_j} \sum_n \omega(y_{i,t-1}) \gamma_{j,n} q_{i,t}
    \end{align*}
    
\end{itemize}

  %  \begin{enumerate}
  %      \item The bins may need to be large to estimate them
  %      \item We may actually want to group by \textit{ex ante} expected score, which we don't have
   % \end{enumerate}


\end{frame}



\begin{frame}{Using Kernel Estimation}

\begin{itemize}
    \item For each teacher: estimate impact of teachers over pretests nonparametrically
     \begin{align*}
    y_{it}  &= g(y_{i,t-1}) + u_{i,t}
    \end{align*}
\item We then get fitted values for teacher $j$, i.e.  $\hat{g}_j(y_{i,t-1})$
\item Then for each teacher we sum their weighted impact on students 

      \begin{align*}
    \hat{W}_j = \sum_{i \in \mathcal{C}_j} \omega(y_{i-1}) \hat{g}_j(y_{i,t-1})
    \end{align*}
    
\end{itemize}

 %       \begin{enumerate}
 %       \item Flexibly estimate teacher impact on students 
 %       \item May be very data hungry; hard to include controls
 %   \end{enumerate}
 
\end{frame}



%
%Quantile


%We don't do this yet
%\begin{enumerate}
%        \item Not sure if the interpretation fits the goals 
%%        \item Still thinking through estimation 
%    \end{enumerate}

%------------------------------------------------
% Testing on Simulated Data 
%-------------------------------------------------
\section{Simulation Analysis}

\begin{frame}{Intro to Simulation}
\begin{itemize}
    \item Start with an illustrative example 
    \item Teachers have a region where they are best at teaching 
    \item Baseline impact is equal for all teachers 
    \item Students are Normally distributed and randomly assigned
    \item 100 teachers, 150 students per teacher, 200 Monte Carlo simulations 
\end{itemize}

\end{frame}

%------------------------------------------------
%plot: Example Teacher Impact
%-----------------------------------------------

\begin{frame}{Expected Teacher Impact is Better for Average Students } 

\begin{figure}
    \centering
\includegraphics[width=.85\linewidth]{slides/CIERS_Figures/average_teacher_example_1.png}
\end{figure}
\end{frame}

%------------------------------------------------
%plot: capturing Teacher impact 
%-----------------------------------------------

\begin{frame}

\frametitle<1>{Each Teacher Has Students They Teach Best}
\frametitle<2>{Traditional Value Added Captures Some Information}
\frametitle<3->{We Can Recover Heterogeneous Value Added}
\vfill
\begin{figure}
    \centering
\includegraphics<1>[width=.85\linewidth]{slides/CIERS_Figures/teacher_example_just_truth_1.png}
\includegraphics<2>[width=.85\linewidth]{slides/CIERS_Figures/teacher_example_truth_standard_1.png}
\includegraphics<3>[width=.85\linewidth]{slides/CIERS_Figures/teacher_example_truth_bin_1.png}
\includegraphics<4>[width=.85\linewidth]{slides/CIERS_Figures/teacher_example_truth_np_1.png}
\includegraphics<5>[width=.85\linewidth]{slides/CIERS_Figures/teacher_example_1.png}
\end{figure}
\end{frame}



%------------------------------------------------
%plot: rawlsian weight (base case)
%-----------------------------------------------

\begin{frame}{From Value Added To Welfare: Rawlsian  Weights}
\vfill
\begin{figure}
    \centering
\includegraphics[width=.85\linewidth]{slides/CIERS_Figures/weight_example_1.png}
\end{figure}
\end{frame}


%------------------------------------------------
%plot: rawlsian caterpillar plots 
%-----------------------------------------------

\begin{frame}
\frametitle<1>{Heterogeneous Value Added Implies Welfare Added}
\frametitle<2>{Standard Value Added Performs Very Poorly}
\frametitle<3>{Nonparametric Value Added Performs Very Well}
\vfill

\begin{figure}
\centering
\begin{subfigure}{.5\textwidth}
  \centering
  \includegraphics<1>[width=\linewidth]{slides/CIERS_Figures/truth_cat_run_2.png}
  \includegraphics<2->[width=\linewidth]{slides/CIERS_Figures/standard_cat_run_2.png}
\end{subfigure}%
\begin{subfigure}{.5\textwidth}
  \centering
  \uncover<3->{
  \includegraphics[width=\linewidth]{slides/CIERS_Figures/ww_cat_run_2.png}}
\end{subfigure}
\end{figure}

\end{frame}

%------------------------------------------------
%figures: rawlsian histogram and sum stats 
%-----------------------------------------------

\begin{frame}{Quantifying the Difference Between Methods} 

\vfill
\begin{figure}
    \centering
    \includegraphics[width=.75\linewidth]{slides/CIERS_Figures/hist_run_2.png}
\end{figure}

\end{frame}


%------------------------------------------------
%plot: plots over teacher center
%-----------------------------------------------

\begin{frame}{In This Example Things Break Because of Implicit Weights}

\vfill

\begin{figure}
\centering
\begin{subfigure}{.5\textwidth}
  \centering
  \includegraphics[width=\linewidth]{slides/CIERS_Figures/standard_cent_run_2.png}
\end{subfigure}%
\begin{subfigure}{.5\textwidth}
  \centering
  \includegraphics[width=\linewidth]{slides/CIERS_Figures/welfare_cent_run_2.png}
\end{subfigure}
\end{figure}

\end{frame}


%-------------------------------------------------
%plot: histogram for more realistic simulations
%-------------------------------------------------

\begin{frame}
\frametitle<1>{Adding Variance in Teacher Ability and Center}
\frametitle<2>{Adding Peer Effects and Student Sorting}
\vfill
\begin{figure}
    \centering
 \includegraphics<1>[width=.75\textwidth]{slides/CIERS_Figures/hist_run_4.png}

 
 \includegraphics<2>[width=.75\textwidth]{slides/CIERS_Figures/hist_run_6.png}
\end{figure}

 \only<1>{
    \hyperlink{bins1}{\beamerbutton{Binned Estimates}}
    \hyperlink{cat1}{\beamerbutton{Caterpillar Plots}}
    \hyperlink{hist8}{\beamerbutton{Variance = 0.5}}
    \hyperlink{hist10}{\beamerbutton{Variance = 1}}
    \hypertarget{run4}{}
 }
 \only<2>{
    \hyperlink{bins2}{\beamerbutton{Binned Estimates}}
    \hyperlink{cat2}{\beamerbutton{Caterpillar Plots}}
    \hypertarget{run6}{}
 }

\end{frame}


%------------------------------------------------------------------------------------------
%------------------------------------------------------------------------------------------
% stress test 
%------------------------------------------------------------------------------------------
%------------------------------------------------------------------------------------------
\section{Stress Test Weighted VAM}


%------------------------------------------------
%plot 
%-----------------------------------------------

\begin{frame}{Traditional Methods Perform Well With More Cross-Teacher Variance}
\vfill
\begin{figure}
\centering
\begin{subfigure}{.5\textwidth}
  \centering
  \includegraphics[width=\linewidth]{slides/CIERS_Figures/stress_bin_ta_sd.png}
\end{subfigure}%
\begin{subfigure}{.5\textwidth}
  \centering
  \includegraphics[width=\linewidth]{slides/CIERS_Figures/stress_np_ta_sd.png}
\end{subfigure}
\end{figure}

\end{frame}

%------------------------------------------------
%plot 
%-----------------------------------------------

\begin{frame}{Our Methods Perform Well With More Within-Teacher Variance}
\vfill
\begin{figure}
\centering
\begin{subfigure}{.5\textwidth}
  \centering
  \includegraphics[width=\linewidth]{slides/CIERS_Figures/stress_bin_max_diff.png}
\end{subfigure}%
\begin{subfigure}{.5\textwidth}
  \centering
  \includegraphics[width=\linewidth]{slides/CIERS_Figures/stress_np_max_diff.png}
\end{subfigure}
\end{figure}

\end{frame}

%-------------------------------------------------------------
\begin{frame}{Measure Welfare Best When Pooling Across Years}
\vfill
\begin{figure}
\centering
\begin{subfigure}{.5\textwidth}
  \centering
  \includegraphics[width=\linewidth]{slides/CIERS_Figures/stress_bin_n_stud.png}
\end{subfigure}%
\begin{subfigure}{.5\textwidth}
  \centering
  \includegraphics[width=\linewidth]{slides/CIERS_Figures/stress_np_n_stud.png}
\end{subfigure}
\end{figure}
    
\end{frame}


%-------------------------------------------------------------
% Did we want a frame showing that once there are a lot of students var(e_it) doesn't matter? We could just say it.


%------------------------------------------------------------------------------------------------------
% CONCLUDING SECTION 
%-------------------------------------------------------------------------------------------------------
\section{Feedback and Next Steps}

%------------------------------------------------
\begin{frame}[label=next]{Next Steps}

\begin{itemize}

    \item More stress testing and experimenting to understand methods 
    
    \item Get data and calibrate simulations
    
    \item Extensions like teacher effort, teacher PPF, long term effects, policy counterfactuals
\end{itemize}

\end{frame}

%------------------------------------------------

\setbeamertemplate{navigation symbols}{}

\begin{frame}[c,noframenumbering]
\centering
\Huge{\centerline{To Be Continued...}}
\normalsize njmather{\fontfamily{qag}\selectfont @}umich.edu \hspace{2em}
ricksmi{\fontfamily{qag}\selectfont @}umich.edu \hspace{2em} \normalsize teastmond{\fontfamily{qag}\selectfont @}ucsd.edu
\end{frame}

%------------------------------------------------


\begin{frame}[noframenumbering]
\frametitle{References}
\tiny
\bibliography{citations}
\end{frame}


%----------------------------------------------------------------------------------------
%----------------------------------------------------------------------------------------

\appendix


%----------------------------------------------------------------------------------------
%----------------------------------------------------------------------------------------



%----------------------------------
% Method Three: Kernel Regression
%------------------------------------

\begin{frame}{Methods: Semiparametric Index Model}

\begin{itemize}

    \item This method flows (most) directly from the SP motivation about weights
    
    \item  Nonparametrically identify teacher $j$'s effects over the achievement distribution
    \begin{itemize}
        \item Estimate $y_i = \sum _j D_j g_j(\beta X_i) +e_i$  by adapting Ichimura's method \citep{ichimura1993semiparametric}
        \item Integrate over the effects using desired welfare weights
    \end{itemize}
    
    \item Intuitive analogue: predict $\hat{y_i}$ then estimate $\mathbb{E}[y_i|\hat{y_i},j=j']$ for each teacher $j'$

    \item Questions/Issues:
    \begin{itemize}
         \item Are there results to estimating subgroup effects?
         \item This will probably only be powered if we pool over years
         \item How do we think about bandwidth selection? 
    \end{itemize}

    %(Also a place to talk about measuring scores in pctle vs $\sigma$?)

\end{itemize}

\vfill
\begin{flushleft}

\hyperlink{next}{\beamerbutton{Back}}
\end{flushleft}
\end{frame}

\begin{frame}{Methods: Lowbel 2015}

\begin{itemize}
\item Let $W_i = (X_i,D_i)$, then 

\begin{align*}
    Y_i &= \sum_k g_k (w_{ki} u_{ki}) +u_{0i} = \sum _j D_{j(i)} g_j(\beta X_i) +u_{0i} \\
        & \iff g_k = 0 \forall k\leq |X_i| \text{ and } g_k(D_{ki} u_{ki}) = D_{j(i)} g_j(\beta X_i) \forall k>|X_i| 
\end{align*}

\item This will be true if $g_j(0)=0$, $u_{ki}=\beta X_i$, and $u_{ki}\neq 0$, but even if that is possible, I don't think $\beta X$ will ever be uncorrelated with all $D_j$?
\end{itemize}


\end{frame}

%-----------------------------------------------------------
%-----------------------------------------------------------

\begin{frame}{Adding Variance in Teacher Ability and Center}

\hypertarget{bins1}{}
\vfill

\begin{figure}
    \centering
 \includegraphics[width=.75\textwidth]{slides/CIERS_Figures/hist_run_3.png}
\end{figure}

\hyperlink{run4}{\beamerbutton{Back}}

\end{frame}


%-----------------------------------------------------------
%-----------------------------------------------------------

\begin{frame}{Adding Variance in Teacher Ability and Center}

\hypertarget{cat1}{}
\vfill
\begin{subfigure}{.5\textwidth}
  \centering
  \includegraphics[width=\linewidth]{slides/CIERS_Figures/standard_cat_run_4.png}
\end{subfigure}%
\begin{subfigure}{.5\textwidth}
  \centering
  \includegraphics[width=\linewidth]{slides/CIERS_Figures/ww_cat_run_4.png}
\end{subfigure}

\hyperlink{run4}{\beamerbutton{Back}}

\end{frame}


%-----------------------------------------------------------
%-----------------------------------------------------------

\begin{frame}{Adding Variance in Teacher Ability and Center}

\hypertarget{hist8}{}
\vfill

\begin{figure}
    \centering
 \includegraphics[width=.75\textwidth]{slides/CIERS_Figures/hist_run_8.png}
\end{figure}

\hyperlink{run4}{\beamerbutton{Back}}

\end{frame}


%-----------------------------------------------------------
%-----------------------------------------------------------

\begin{frame}{Adding Variance in Teacher Ability and Center}

\hypertarget{hist10}{}
\vfill

\begin{figure}
    \centering
 \includegraphics[width=.75\textwidth]{slides/CIERS_Figures/hist_run_10.png}
\end{figure}

\hyperlink{run4}{\beamerbutton{Back}}

\end{frame}


%-----------------------------------------------------------
%-----------------------------------------------------------

\begin{frame}{Adding Peer Effects and Student Sorting}

\hypertarget{bins2}{}
\vfill

\begin{figure}
    \centering
 \includegraphics[width=.75\textwidth]{slides/CIERS_Figures/hist_run_5.png}
\end{figure}

\hyperlink{run6}{\beamerbutton{Back}}

\end{frame}


%-----------------------------------------------------------
%-----------------------------------------------------------

\begin{frame}{Adding Peer Effects and Student Sorting}

\hypertarget{cat2}{}
\vfill
\begin{subfigure}{.5\textwidth}
  \centering
  \includegraphics[width=\linewidth]{slides/CIERS_Figures/standard_cat_run_6.png}
\end{subfigure}%
\begin{subfigure}{.5\textwidth}
  \centering
  \includegraphics[width=\linewidth]{slides/CIERS_Figures/ww_cat_run_6.png}
\end{subfigure}

\hyperlink{run6}{\beamerbutton{Back}}
    
\end{frame}

\end{document}

