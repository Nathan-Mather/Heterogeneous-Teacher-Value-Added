\documentclass[12pt]{article}

% import packages for general latex 
\usepackage{imports}


% mike packages 

\usepackage[utf8]{inputenc}
\usepackage{geometry,ulem,graphicx,caption,color,setspace,dsfont,physics,commath,amsfonts,bm}

\usepackage{caption}
\usepackage{subcaption} 
\usepackage[short]{optidef}
\usepackage{hhline}
\usepackage[capposition=top]{floatrow}
\usepackage{booktabs} % Allows the use of \toprule, \midrule and \bottomrule in tables
\usepackage{adjustbox}
\usepackage{tikz}
\usepackage{pdflscape}
\usepackage{afterpage}
\usetikzlibrary{calc,patterns,positioning}
\usepackage{environ}
\usepackage{natbib,hyperref}
\usepackage{soul}

\usepackage[amsthm]{ntheorem}
\hypersetup{ hidelinks }

\theoremstyle{definition}
\newtheorem{innercustomthm}{Assumption}
\newenvironment{customthm}[1]
  {\renewcommand\theinnercustomthm{#1}\innercustomthm}
  {\endinnercustomthm}

\theoremstyle{definition}
\newtheorem{assumption}{Assumption}


\theoremstyle{definition}
\newtheorem{auxa}{Aux. Assumption v}

\theoremstyle{definition}
\newtheorem{definition}{Definition}
\newtheorem{thm}{Theorem}


\newcommand*\diff{\mathop{}\!\mathrm{d}}
%\DeclareMathOperator*{\argmax}{arg\,max}
%\DeclareMathOperator*{\argmin}{arg\,min}


\usepackage{titlesec}
\titleformat{\section}
  {\normalfont\normalsize\bfseries}{\thesection.}{1em}{}

\titleformat{\subsection}
  {\normalfont\normalsize\bfseries}{\thesubsection}{1em}{}

%% For editing
\newcommand\cmnt[2]{\;
{\textcolor{red}{[{\em #1 --- #2}] \;}
}}
\newcommand\nate[1]{\cmnt{#1}{Nate}}
\newcommand\rmk[1]{\;\textcolor{red}{{\em #1}\;}}
\newcommand\natenote[1]{\footnote{\cmnt{#1}{Nate}}}


\makeatletter
\newsavebox{\measure@tikzpicture}
\NewEnviron{scaletikzpicturetowidth}[1]{%
  \def\tikz@width{#1}%
  \def\tikzscale{1}\begin{lrbox}{\measure@tikzpicture}%
  \BODY
  \end{lrbox}%
  \pgfmathparse{#1/\wd\measure@tikzpicture}%
  \edef\tikzscale{\pgfmathresult}%
  \BODY
}
\makeatother


\DeclareCaptionLabelFormat{AppendixTables}{A.#2}



\title{From Value Added to Welfare Added: \\ A Social-Planner Approach Applied to Education Policy}

\author{Tanner S. Eastmond\thanks{Department of Economics, University of California, San Diego: \texttt{teastmond@ucsd.edu}, \texttt{jbetts@ucsd.edu}} \and Nathan J Mather\thanks{Department of Economics University of Michigan: \texttt{njmather@umich.edu}} \and Michael David Ricks\thanks{National Bureau of Economic Research: \texttt{ricksmi@umich.edu} \hspace{17em} {\color{white}t} This research is the product of feedback and from many people including Ash Craig, Jim Hines, Gordon Dahl, Lars Lefgren, Peter Hull, Nathan Hendren, Andrew Bacher-Hicks, William-Delgado, Amy Finklestein, %Jesse Rothstein,
Andrew Simon, and  researchers at the Education Policy Initiative, Youth Policy Lab, and SANDERA as well as with seminar participants at the University of California - San Diego, the University of Michigan, Brigham Young University, CAL Labor Summit, and Boston University. Thanks also to Andy Zau who facilitated the data access and to  Wendy Ranck-Buhr, Ron Rode, and others at the San Diego Unified School District for their interest and feedback.} \and Julian Betts$^{*\ddagger}$}


\geometry{left=1.0in,right=1.0in,top=1.0in,bottom=1.0in}

% Start of the document 
\begin{document}
\maketitle


%We use welfare theory to recover normatively relevant information from heterogeneous and multidimensional estimates of value added and consider optimal policy of allocating teachers to classes.

\begin{abstract}

Though ubiquitous in research and practice, mean-based “value-added” measures may not fully inform policy or welfare considerations when policies have heterogeneous effects, impact multiple outcomes, or seek to advance distributional objectives. In this paper we formalize the importance of heterogeneity for calculating social welfare and quantify the importance of heterogeneity in an enormous public service provision problem: the allocation of teachers to elementary school classes. Using data from the San Diego Unified School District we estimate heterogeneity in teacher value added over the student achievement distribution. Because a majority of teachers have significant comparative advantage across student types, allocations that use a heterogeneous estimate of value added can raise scores by 34-97\% relative to those using standard value added estimates. These gains are even larger if the social planner has heterogeneous preferences over groups. Because reallocations benefit students on average at the expense of teachers' revealed preferences, we also consider a simple teacher compensation policy, finding that the marginal value of public funds would be infinite for bonuses of up to 16\% of baseline pay. These results, while specific to the teacher assignment problem, suggest more broadly that using information about effect heterogeneity might improve a broad range of public programs—both on grounds of average impacts and distributional goals.

\end{abstract}

\onehalfspacing

\pagebreak 

\section{Introduction}

% Paragraph(s) about why we care: 
%    -Value added is common, 
%    -heterogeneity won't be captured in these means
%    -Research questions: when does het matter? How bad is ignoring it in assigning teachers to classrooms?

% set up and what we do in this paper 
When evaluating policies, programs, and institutions researchers often rely on mean impacts. While means are powerful summary measures, they can also mask economically important information. This paper seeks to understand how measuring heterogeneity can more fully inform welfare measures and better optimize policy choices. We ask two main questions. (1) Theoretically, when does heterogeneity (in effects, outcomes, and social preferences) matter for maximizing a social objective? (2) Empirically, how large are the welfare gains from using heterogeneous rather than average estimates of impacts to evaluate and refine public policy?


% Wondering if this should be one or two paragraphs?
Although these questions have many applications, we explore them in the context of value added scores for elementary school teachers. Many have used value added scores (regression adjusted means) to measure the effects of teachers and schools \citep[see reviews in][]{angrist2022methods,bacher2022estimation}; doctors, hospitals, and nursing homes \citep{chandra2016health,doyle2019evaluating,hull2020hosptial,einav2022producing,chan2022selection}; and even judges, prosecutors and defense attorneys \citep{abrams2007luck,norris2019examiner,harrington2023prosecutor}. We choose this setting because of mounting empirical evidence that value added scores are both \textit{multidimensional} and \textit{heterogeneous} in the education context. For example, teachers affect student outcomes in multiple dimensions such as math and reading scores \citep{condie2014teacher}, attendance and suspensions \citep{jackson2018test}, and work ethic and learning skills \citep{pope2017multidimensional}. Furthermore, teachers also have heterogeneous effects on different types of students defined by factors such as race and gender \citep[e.g.,][]{dee2005teacher,delhommer2019highschool,Delgado2020} and socioeconomic status \citep{bates2022teacher}. Similar patterns have been found in health-related value added \citep[e.g.,][]{hull2020hosptial,amyspaperwhenitcomesout}. 

This paper applies and extends insights from theoretical welfare economics to overcome the limitations that arise from multidimensionality and heterogeneity, allowing us to empirically evaluate the optimal allocation of teachers to classes based on this information.  The critical issue from a social welfare perspective is that in the presence of multidimensionality and heterogeneity value-added measures only partially order the welfare of an allocation of teachers to students.  Intuitively, this is because of ambiguity about whether the definition of a ``better'' teacher should prioritize gains in math versus reading scores or gains for high-achieving versus low-achieving students  \citep[see impossibility-like results in][]{condie2014teacher}. Fortunately, whereas research in value added has identified these problems, research in public finance has a long history of using using welfare functions to aggregate over the heterogeneous effects of policies. We apply and extend such insights from welfare economics for two purposes. First, we  characterize the shortcomings of relying on mean-oriented measures of policy effects such as standard value added to make welfare considerations in general. Then the bulk of the paper evaluates the optimal allocation of teachers to classes in particular using measures of heterogeneous value added that produce scalar, welfare-relevant statistics.

Our theoretical results show two ways that ignoring effect heterogeneity can lead to inaccurate inference about both policy counterfactuals and how policy can be improved. First, bias arises when means effects are not externally valid to match effects from the policy. For example, imagine a medical treatment that did not have serious side effect in the population in general. If we are considering a policy that would target this treatment to new high-risk patients, it is not clear whether the impact will be the same. Second, bias also arises from the covariance across the target population of the effects of a policy and and welfare weights. For example, consider a tax reform that raises post-tax incomes by \$3000 to the richest 50\% of households but reduces incomes by \$1000 for the poorest 50\% of households. Policymakers may consider this reform undesirable for equity reasons even though it increases average incomes. These biases can both be reduced or eliminated by estimating conditional average treatment effect along appropriate observable dimensions and allowing for heterogeneous welfare weights. When optimizing policy, correcting this bias can lead to significant gains through comparative advantage and allow policymakers to direct interventions towards people with the highest marginal welfare benefit. 
 

% These examples are general 
%Both of the above motivations for measuring heterogeneity have broad applications beyond these specific examples. The economic literature most often considers heterogeneity and welfare weights in the context of redistributing dollars (the optimal tax literature for example). However, the same dynamics can be at play for any mean outcome of interest. In addition to income, policymakers care about things like test scores, life expectancy, fitness, nutrition, or employment; moreover, they often care about \textit{who} receives the gains in those outcomes and who, if anyone, loses. Additionally, assigning teachers to new classrooms is not the only case where measuring heterogeneity will improve counterfactual estimates. Other policies assigning practitioners like doctors, judges, police, or even firms to different heterogeneous populations have the same dynamics.

These theoretical results highlight an interesting contribution of our paper. As empirical policy evaluations become increasingly common, our theoretical results characterize the trade-offs implicit in relying on mean impacts. For example, using mean effects to predict the welfare of an allocation is biased in general because welfare depends not just on program impacts and welfare weights but the covariance of the two. Interestingly, this insight is reminiscent of similar results in optimal corrective taxation of heterogeneous consumption externalities (like alcohol). \citet{griffith2019tax} show that the optimal corrective tax is the average consumption externality \textit{plus} the covariance between individual contributions to the externality (the effect) and demand elasticities (the weight). Furthermore, in the externality context, conditioning (in this case tax differentiation by product) also reduces the bias, as it can in our setting.\footnote{The second insight is technically a generalization of the first, which was originally suggested in \citet{diamond1973consumption}.} The importance of heterogeneity and conditioning in these theoretical settings raises questions about whether using average ``sufficient statistics'' is appropriate when heterogeneous estimates could inform differentiated policies like corrective taxation of heterogeneous \textit{production} externalities \citep[][]{hollingsworth2019external,fell2021emissions,sexton2021heterogeneous}, [2], [3]. Crucially, we speak to these trade-offs by showing how both biases can be reduced by estimating conditional average treatment effects along observable dimensions to allow for heterogeneity in impacts.


Motivated by the importance of heterogeneity in general, we estimate heterogeneity in teacher value added along the achievement distribution in the San Diego Unified School District, the second largest district in California. We find large gains from using heterogeneity to more optimally allocate teachers to students. We use the methods pioneered by \citet{Delgado2020} to estimate the value added of all third- through fifth-grade teachers on student math and English language arts (ELA) scores allowing for heterogeneous effects on students who had above- and below-median scores the previous year. Although these measures of value added are correlated with standard (i.e. homogeneous value added) measures, we find substantial heterogeneity. For example, the average within-teacher difference in value added  across groups (i.e., comparative advantage) is as large as 53\% (48\%) of a standard deviation in mean value added for ELA (math). We use these estimates to consider welfare gains from two sets of possible policies: reallocating teachers to classes without changing school assignment or allowing for school reassignment.\footnote{In all reallocations the assignment of students to classes is held constant, as is the grade in which the teacher teaches.} There are enormous gains from reallocation. Over the course of third to fifth grade, using heterogeneous measures of value added to improve district-wide teacher assignments could raise student math scores by 0.17 student standard deviations on average and ELA scores by 0.12. For context, both changes are roughly equivalent to an intervention improving all teachers value added by 30\% of the (teacher) standard deviation in the relevant subject.

In this process, our paper makes three innovative contributions to the literatures on value added and teacher value added. First, we demonstrate how important achievement is as a dimension of effect heterogeneity in our education context. Whereas many papers have found evidence of ``match effects'' between students and teachers sharing  observable characteristics like gender or race \citep{dee2005teacher,delhommer2019highschool}, other results reveal that these match effects only explain part of the heterogeneity in teacher effects on the same dimensions \citep{Delgado2020}. Our results suggest that focusing on demographic match is incomplete because it overlooks how instructional differentiation along the achievement distribution \citep[well documented in the education literature][]{} interacts with these correlated with these characteristics. This insight reflects other evidence from health economics that in general lagged outcomes are one of the most important dimensions for match effect heterogeneity \citep[as in][]{Dahlstrand2022defying}.

Second, our results highlight the importance of combining information from multiple outcomes substantially improves the welfare gains from reallocations. Although it is not obvious \textit{ex ante} how to address this multidimensionality, our theory suggests combining outcomes based on how they affect long-term outcomes of interest. To this end, we aggregate teacher effects using estimates of the differential impact of elementary school gains in math and ELA on lifetime earnings from \citet{chetty2014measuring2}. Back of the envelope calculations suggest that over three years the allocation of teachers that maximizes present-valued lifetime earnings would generate over \$4000 in present valued earnings per student or over \$83.7 million in total.\footnote{Here present valuation is discounted at 3\% following back to age 10 following \citet{krueger1999experimental} and \citet{chetty2014measuring2}.} Whereas interventions in the education literature have often focused on math scores for a variety of reasons \citep{chetty2014measuring1,Delgado2020,bates2022teacher}, our contribution is accounting for the separate marginal effects of math and reading outcomes, which generates 34\% larger wage impacts (value added of \$21 million) relative to focusing only on math. %Sentence about this in other contexts

Third, these results have implications for the discussion of using value added in teacher (and doctor and hospital) compensation and extend our understanding of the welfare implications of such policies. Motivated by the large earnings gains from reallocations, we explore the welfare implications of using lump-sum transfers to compensate teachers for the possibility of being reallocated. We calculating the marginal value of public funds \citep[MVPF][]{Keyser_2020} for varying sizes of bonus payments to all teachers and find enormous gains. The MVPF of bonuses in the disctrict-wide reallocation is infinite up to for up to \$8300 per teacher (16\% of the median income in the district). For within-school-grade reallocations---which have smaller gains but which should be all but costless to teachers---we find that the MVPF is infinite for bonuses of up to \$2200.  These insights combines insights from two literatures on teacher labor markets: one focusing on dismissal \citep{hanushek2009teacher,staiger2010searching,chetty2014measuring1}, but sometimes ignoring teacher supply decisions \citep[as pointed out in][]{rothstein2010teacher} and the other characterizing teacher demand \citep{Johnson2021preferences} but sometimes ignoring teacher effect \citep[as addressed in][where both are combined]{bates2022teacher}. Our contribution is characterizing the welfare effects of policies that use teacher value added but compensate teachers for the possible disutility of the resulting allocation.


Taken together, our results highlight the first-order importance of considering heterogeneity in empirical welfare analysis. In our theory we show how the gains possible from allocations based on heterogeneous effects may be much larger than those based on means only. We document this empirically in our setting where considering just one dimension of heterogeneity increase test score gains by 34-97\% relative to only using the standard value added measure. While the critical role of comparative advantage has been acknowledged for centuries, our contribution to welfare theory is in connecting treatment effect heterogeneity, comparative advantage, and social preferences. These connections capture and formalize the growing understanding that heterogeneity is a key consideration for allocating scarce resources according to a social objectives by means of targeting---as has been explored theoretically \citep{kitagawa2018should,athey2021policy} and as is reflected in a recent explosion of empirical inquiry about targeting treatments as varied as social safety programs \citep{alatas2016self,finkelstein2019take}, costly energy efficiency interventions \citep{ito2021selection,ida2022choosing} and even resources to reduce gun violence \citep{bhatt2023predicting} or promote entrepreneurship in developing countries \citep{hussam2022targeting}. Our results suggest that in these settings and others ignoring heterogeneity may have serious welfare ramifications and that considering heterogeneity in effects and social preferences presents a clear path forward for future welfare analyses.

%This suggests enormous gains from reallocating teachers to classes with more students they have a comparative advantage in teaching. We also thoroughly document the reliability of these estimates by demonstrating that they are forecast unbiased (up to NNN\%), persistent over time (with year-to-year correlations between 0.78-0.90), and equally predictive of later-life outcomes as standard value added (despite being estimated on half the sample).











% Not sure how to build in the explicit results rather than just the relative gains, worth trying though.

% First, our results show enormous gains from making allocations based off of heterogeneous estimates of value added. We find that a social planner seeking to maximize average scores could increase both lower- and higher-scoring students' ELA scores by 0.04 student standard deviations. Gains from math are even larger: 0.04 for lower-scoring students and 0.07 for higher. Compared to an allocation that used standard value added measures to allocate the best teachers to the largest classes, our reallocation generates 34-97\% higher gains in test scores. An interesting implication of our results is that by reallocating the traditional assumptions of constant effects and equal class size we can reallocate rather than release teachers. ``Deselecting'' low-performing teachers has been the traditional policy counterfactual in value added papers for a quarter of a century. However in our setting deselection would only raise average math scores by 0.021 $\sigma$---a smaller figure than  simply putting better teachers in larger classes would create, and one that is dwarfed by reallocations using comparative advantage 0.070. This highlights the critical role of comparative advantage in welfare considerations. 

%Furthermore, deselection using standard value added penalizes teachers who happen to be allocated to worse-matched classes generating a 16-19\% discrepancy between rankings in the lowest five percent.



% Paragraph(s) about what we do - Welfare Students (het/multidimensionality)
%    - 
%    - 
%    - 

% Paragraph(s) about what we do - Welfare Teachers






%Dalhstrom ppaper on match effects of lag outocmes.


This paper is organized into 6 sections. Section 2 introduces our framework for welfare and value added with the implications of heterogeneity. Section 3 contains our estimation procedure and a description of value added in the San Diego Unified School District. Section 4 leverages our welfare theory to explore the reallocation of teachers to classes and measures the welfare gains from using information about heterogeneity. Finally, Section 5 draws the pieces together to explore the implications for welfare and Section 6 concludes.



\bibliography{citations}

\end{document}