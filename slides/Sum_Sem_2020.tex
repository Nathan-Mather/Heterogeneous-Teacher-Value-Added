%%%%%%%%%%%%%%%%%%%%%%%%%%%%%%%%%%%%%%%%%
% Beamer Presentation
% LaTeX Template
% Version 1.0 (10/11/12)
%
% This template has been downloaded from:
% http://www.LaTeXTemplates.com
%
% License:
% CC BY-NC-SA 3.0 (http://creativecommons.org/licenses/by-nc-sa/3.0/)
%
%%%%%%%%%%%%%%%%%%%%%%%%%%%%%%%%%%%%%%%%%

%----------------------------------------------------------------------------------------
%	PACKAGES AND THEMES
%----------------------------------------------------------------------------------------
\documentclass[t,aspectratio=169,11pt]{beamer}
%\documentclass[t,aspectratio=169,11pt,draft]{beamer}

\mode<presentation> {

\input{LatexColors.incl.tex}

\usetheme{Boadilla}

% Footer and page numbers

% First remove the navigation symbols from the bottom of all slides 
\setbeamertemplate{navigation symbols}{} 
% Make a new prettier page number
\addtobeamertemplate{navigation symbols}{}{%
    \usebeamerfont{footline}%
    \hspace{5em}%
    {\color{black!50}{\small\insertframenumber/\inserttotalframenumber}}%
}
% Remove the colored line at the bottom
\setbeamertemplate{footline}{}


% Customize items
\setbeamertemplate{itemize items}{\color{ptlightblue}{\rule[0.05em]{0.6em}{0.6em}}}
\setbeamertemplate{itemize subitem}{\color{ptlightblue}{\raisebox{0.12em}{$\blacktriangleright$}}}
\setbeamercolor{enumerate item}{fg=black}
\setbeamertemplate{enumerate items}[default]
\setbeamercolor{enumerate subitem}{fg=black}
\setbeamertemplate{enumerate subitems}[default]


% Table of Contents (if we have one)
\setbeamertemplate{section in toc}{\inserttocsectionnumber.~\inserttocsection}
\setbeamercolor{section in toc}{bg=white,fg=black}
\setbeamertemplate{subsection in toc}{%
  \hspace{1.2em}{{\color{ptlightblue}{\rule[0.05em]{0.6em}{0.6em}}}~\inserttocsubsection\par}
}
\setbeamercolor{subsection in toc}{bg=white,fg=black}


% Customize beamer colors
\setbeamercolor{titlelike}{fg=white,bg=ptlightblue}
\setbeamercolor{button}{bg=ptlightcyan,fg=white}
}

% Nicer looking Itemize and enumerate environments
\newenvironment{wideitemize}{\itemize\addtolength{\itemsep}{14pt}}{\enditemize}
\newenvironment{wideenumerate}{\enumerate\addtolength{\itemsep}{14pt}}{\endenumerate}

% Import packages
\usepackage[default]{lato} % More modern looking font than Helvetica
\usepackage{geometry,ulem,caption,color,setspace,dsfont,physics,commath,amsmath,amssymb}
\usepackage[comma]{natbib}
\usepackage[short]{optidef}
\usepackage{hhline}
\usepackage{array}
\usepackage{booktabs} % Allows the use of \toprule, \midrule and \bottomrule in tables
\usepackage{adjustbox}
\usepackage{graphicx,xcolor,bbm,xcomment}
\usepackage{appendixnumberbeamer}
\usepackage{textcomp}
\usepackage{colortbl}
\usepackage{subcaption} 
\usepackage{tikz}
\usetikzlibrary{calc,patterns,positioning}
\usepackage{pdflscape}
\usepackage{environ}

\bibliographystyle{ecta}



% Make Section Header Frames
\AtBeginSection[]{
  \begin{frame}
  \vfill
  \centering
  \begin{beamercolorbox}[sep=8pt,center,shadow=true,rounded=true]{title}
    \usebeamerfont{title}\insertsectionhead\par%
  \end{beamercolorbox}
  \vfill
  \end{frame}
}

\AtBeginSubsection[]{
  \begin{frame}
  \vfill
  \centering
  \begin{beamercolorbox}[sep=8pt,center,shadow=true,rounded=true]{title}
    \usebeamerfont{title}\insertsectionhead: \\ \insertsubsectionhead\par%
  \end{beamercolorbox}
  \vfill
  \end{frame}
}



\tikzset{
    invisible/.style={opacity=0},
    visible on/.style={alt={#1{}{invisible}}},
    alt/.code args={<#1>#2#3}{%
      \alt<#1>{\pgfkeysalso{#2}}{\pgfkeysalso{#3}} % \pgfkeysalso doesn't change the path
    },
  }



%----------------------------------------------------------------------------------------
%	TITLE PAGE
%----------------------------------------------------------------------------------------

\title{From Value Added to Welfare Added} 

\author{Tanner Eastmond \hfill \and Nathan Mather   \and \hfill Michael D Ricks   \\   \textit{teastmond@ucsd.edu}\hfill \textit{njmather{\fontfamily{qag}\selectfont @}umich.edu} \hfill
\textit{ricksmi{\fontfamily{qag}\selectfont @}umich.edu} } 
\institute{}
\date{\today} 

\begin{document}

\begin{frame}
\titlepage 
\end{frame}

%------------------------------------------------
% What is Value added 
%------------------------------------------------
\begin{frame}{Value Added Measures: The Goal}
\begin{wideitemize}
    \item Value added measures (VAM) identify the most and least effective teachers
    \item We know teachers respond to incentives
    \begin{itemize}
        \item {\color{gray}{\citep{jacob2003rotten,neal2010left,pope2017multidimensional}}}
    \end{itemize}
    \item We know VAM capture something real
    \begin{itemize}
        \item {\color{gray}{\citep{chetty2014measuring2,pope2017multidimensional}}}
    \end{itemize}
    \item We know there is heterogeneity VAM do not catch
    \begin{itemize}
        \item {\color{gray}{\citep{ehrenberg1995teachers,dee2005teacher}}}
    \end{itemize}
\end{wideitemize}

\end{frame}

%------------------------------------------------
% Why do we care 
%------------------------------------------------
\begin{frame}{Value Added Measures: A Policy Disconnect  }
\begin{wideitemize}

    \item Disconnect between VAM and education policy such as No Child Left Behind
    \begin{itemize}
        \item VAM are mean-oriented statistics: a teachers average impact on students' scores
        \item Carries \textbf{implicit normative} welfare weights 
    \end{itemize}
    
    \item  We propose a set of more flexible heterogeneous VAM 
    \begin{itemize}
        \item Assess teachers' heterogeneous impacts
        \item  Weight effects according to an \textbf{explicit normative} policy goal or welfare criterion
    \end{itemize}


\end{wideitemize}

\end{frame}

%------------------------------------------------
% How do we answer our question?
%------------------------------------------------

\begin{frame}{Research Goals}

\begin{wideenumerate}
     \item Build a welfare-relevant framework for VAM (and other educational statistics) 
     \begin{itemize}
         \item Can we accommodate any policy or normative goal?
     \end{itemize}
    \item Estimate VA heterogeneity along the achievement distribution
    \begin{itemize}
        \item Do certain teachers excel at teaching certain students? 
    \end{itemize}
    \item Simulate data to test new methods
    \begin{itemize}
        \item Are these methods feasible? Accurate? Informative? 
    \end{itemize}
    \item Employ methods in real data
    \begin{itemize}
        \item What implications do rank inversions have for welfare?
        \item What implications are there for possible policies?
    \end{itemize}  

\end{wideenumerate}
\end{frame}

%-----------------------------------------------
% Goal of toady's talk 
%-----------------------------------------------
\begin{frame}[c]{Today's Talk}

\begin{wideenumerate}
    \item Big picture, does this look like it is worth continuing?
    \item Which of our possible next steps are most worth pursuing?
    \item What first order concerns should we be aware of and thinking about?

\end{wideenumerate}
\end{frame}




%------------------------------------------------------------------------------------------
%------------------------------------------------------------------------------------------
% NEW SECTION: Setting up the theory 
%------------------------------------------------------------------------------------------
%------------------------------------------------------------------------------------------
\section{Welfare and Value Added}

%------------------------------------------------
% Estimating STANDARD Value Added 
%------------------------------------------------

\begin{frame}{Estimating Standard  Value Added}

\begin{wideitemize}
    \item In theory test scores are a function of testing ability, value added, and noise
    \begin{align*}
    y_{ijt}  &= a_{it} + \epsilon_{ijt} \\
    y_{ijt}  &= a_{it-1} + VA_j(a_{it-1}) + \epsilon_{ijt}
    \end{align*}
 
    \item We don't observe testing ability. Traditional VAM assumptions:
    \begin{itemize}
        \item Predict testing ability with characteristics (including lagged scores): $X_{it}$
        \item Assume homogeneity within and across teachers $VA_j(\cdot)=\gamma_j$ with no $ij$ error
    \end{itemize}
    \begin{align*}
    y_{ijt}  &= \beta X_{it} +\Gamma D_{it} +\mu_{jt} + \eta_{it}
    \end{align*}
    
    \item Note the $\mu_{jt}$ variation could pick up $ij$ variation.
       
    
\end{wideitemize}
\end{frame}

%------------------------------------------------
% Obligatory model going over the logic 
%------------------------------------------------
\begin{frame}{Modeling Value Added As Social Welfare}

\begin{wideitemize}
    \item We can think of the following social welfare function
    \begin{itemize}
        \item $\omega(x_i,y_i)$ weights: likely based on \textit{ex ante} expected performance
        % Talk about the concavity here?
        \item $v(x_i,y_i)$ value: think achievement, gains, etc.
    \end{itemize}
    \[
    W  = \sum_i \omega(x_i,y_i) v(x_i,y_i) 
    \] 
    
    \item Traditional VAM take the average gains for students among each teacher
    \begin{itemize}
        \item Let $x_i$ be \textit{ex ante} expected performance, estimated as $\hat{y}_i$, then $\tilde{v}(\cdot) = y_i - \hat{y}_i$
    \end{itemize}
    \[
    \hat{W}_{VA}  = \sum_i \frac{(y_i-\hat{y}_i)}{N_{j(i)}} \hspace{3em}
    \]
    
    \item This implies $\omega(x_i,y_i)=\frac{1}{N_j}$, which is (almost) utilitarian
    % this also implies that the value we care about is growth above/compared to similar students. What I mean is theoretically the same measures could be achieved by changing V or W, but changing W seams easier.  
    % Not what we hear people talking about 
    % For reference a VA version of NCLB has a weight  .... 
    
    
    % Note VA scores are mean zero (almost--because you are averaging over classes of differnt sizes) , so total welfare is uninformative but rank is informative.
\end{wideitemize}


\end{frame}


%------------------------------------------------
% Obligatory model going over the logic 
%------------------------------------------------
\begin{frame}{``Welfare Added''}

\begin{wideitemize}
    \item Let $\hat{v}_j(x_i,y_i)$ estimate a teacher's heterogeneous value added, $\mathbb{E}[v(\cdot)|x_i,j]$
    
    \item Then the estimated ``Welfare Added'' by this teacher is  
    \[
    \hat{W}_j  = \sum_{i\in j} \omega(x_i,y_i) \hat{v}_j(x_i,y_i) 
    \] 
    
    \item Consider the following examples:
    \begin{itemize}
        \item Utilitarian: All students are weighted equally $\omega(x_i,y_i) = 1$
        \item Rawlsian: We care only about students below some cutoff, $c$, $\omega(x_i,y_i) = \mathds{1}(x_i\leq c)$
        \item Pareto: We care about the lower achieving students more $ \omega(x_i,y_i)=  \frac{\alpha x_\mathrm{m}^\alpha}{x_i^{\alpha+1}}$ 
    \end{itemize}
    
    \item But estimating Welfare Added requires an estimate of $\mathbb{E}[v(\cdot)|x_i,j]$
\end{wideitemize}


\end{frame}

%-----------------------------------------------
% Motivation for Weighted VAM 
%-----------------------------------------------

\begin{frame}{Estimating Effect Heterogeneity}

\begin{wideitemize}
    \item Intuitively, imagine a pointwise approach for estimating $\hat{v}_j(x_i,y_i)$
    
    \item Binning students by \textit{ex ante} expected score and estimating VA has two problems
    \begin{enumerate}
        \item We don't have infinite data, so bins would be too small
        \item We don't actually know the \textit{ex ante} expected score
    \end{enumerate}

    \item $\hat{y}_{it} = \hat{\beta} X_{it} $ consistently predicts expected score, but with error
    \begin{itemize}
        \item The distribution of $\hat{y_i}|x_i$ will be centered at and decreasing from $x_i$
    \end{itemize}

%+ \sum \hat{\beta}_jy_{it-j} because we define X as including lagged scores

    \item We propose three methods that (not only deal with but) capitalize on this
    \begin{enumerate}
        \item WLS regression with weights defined by the Exogenous policy preference
        \item Quantile regression evaluated at the relevant percentile
        \item Kernel regression of effect of each teacher over achievement distribution
    \end{enumerate}

\end{wideitemize}

\end{frame}


%------------------------------------------------------------------------------------------
%------------------------------------------------------------------------------------------
% Going through our methods 
%------------------------------------------------------------------------------------------
%------------------------------------------------------------------------------------------
\section{Measuring Heterogeneity in Teacher Effects}


%-------------------------------------------------------------
% Estimating Weighted Value added with our first method: set up
%-------------------------------------------------------------
\begin{frame}{Weighted VAM: Using Weighted Least Squares}

\begin{wideitemize}

    % set up the method 
    \item We want to estimate coefficients of pretest and other covariates like normal 
    \item Partial out covariates in unweighted regression
    \item Want to weight the impact of students on teachers
    \item Estimate effects of teachers and weight prediction by achievement
        \begin{itemize}
        \item  WLS is consistent in models ``fully saturated'' with binary  variables  \citet{solon2015we} 
        \end{itemize}

\end{wideitemize}



\end{frame}


%-------------------------------------------------------------
% Estimating Weighted Value added with our first method
%-------------------------------------------------------------

%% It might be easier to not call this method one because it adds in a step of complication with having to do the Frisch-Waugh-Lovell projection.
\begin{frame}{Weighted VAM: Using Weighted Least Squares}

\begin{wideitemize}
 \item Let Y be tests, X, controls including pretest, and D a matrix of student-teacher links
\item  Traditional Value Added:
    \begin{align*}
    Y &= \beta X + \Gamma D + \eta \\
    (\hat{\beta}_{OLS},\hat{\Gamma}_{OLS}) &= \left ((X,D)'(X,D) \right ) ^{-1} (X,D)' Y 
    \end{align*}
\item Let $\tilde{D}$ be residuals from regression of D on X \citep{frisch1933partial}
\[
\hat{\Gamma}_{OLS} = \hat{\Gamma}_{FWL} =  (\tilde{D}'\tilde{D}  ) ^{-1} \tilde{D}' Y
\]
\item Weighted estimate of welfare added, given weights $W_p$, is
\[
\hat{\Gamma}_{p} = \left (\tilde{D}'W_p\tilde{D} \right ) ^{-1} \tilde{D}'W_p Y
\]


\end{wideitemize}

\end{frame}





%------------------------------------------------
% Testing on Simulated Data 
%-------------------------------------------------
\begin{frame}{Testing on Simulated Data  }


\begin{wideitemize}
\item Students have a testing ability $\alpha_{it} \sim \mathcal{N}(0,1)$
\item Teachers have relative teaching ability  $\gamma_{j} \sim \mathcal{N}(0,0.1)$
\item Teachers have an area where they are most effective $C_{j} \sim  \mathcal{N}(0,1)$
\item Their impact drops off at rate .15 (capped at .3) as they move away from $C_{j}$
\item Teachers impact student testing ability such that: 
\begin{align*}
\alpha_{it+1} &= \alpha_{it} + \gamma_{j} - 0.15\min(|\alpha_{it} - C_{j}|, 2)  + \epsilon_{it} \\
\epsilon_{it} &\sim  \mathcal{N}(0,0.1) 
\end{align*}


\item Tests distributed as $ T_{it} \sim \mathcal{N}(\alpha_{it},0.07)$
\end{wideitemize}
\end{frame}

%------------------------------------------------
%plot: Example Teacher Impact
%-----------------------------------------------

\begin{frame}{Example Teacher Impact} 

\centering

\includegraphics[width=.85\linewidth]{slides/Figures/teacher_impact_kernal_noise.png}

\end{frame}

%------------------------------------------------
%plot: Average Teacher Impact
%-----------------------------------------------

\begin{frame}{Average Teacher Impact}

\centering

\includegraphics[width=.85\linewidth]{slides/Figures/Average_teacher_impact_kernal.png}

\end{frame}

%------------------------------------------------
% Monte Carlo 
%-----------------------------------------------
\begin{frame}{Monte Carlo Simulation: Calibration}
\begin{wideitemize}
\item Test the performance of our welfare added estimates against Standard VA
\item Metrics compared to True Welfare Weighted teacher impact
\begin{itemize}
        \item Teacher impact times weight using simulated student ability (rather than test 1)
\end{itemize}
\item 10,000 simulations 
\item 100 students per teacher 
\item 132 teachers 
\item 13,200 students
\end{wideitemize}
\end{frame}


%------------------------------------------------
%plot: Linear weight
%-----------------------------------------------

\begin{frame}{Linear Weights} 

\centering

\includegraphics[width=.85\linewidth]{slides/Figures/linear_weight.png}

\end{frame}

%------------------------------------------------
%plot: Linear caterpillar plots 
%-----------------------------------------------

\begin{frame}{Linear Weights: Welfare Added Estimates} 

\centering
 \includegraphics[width=0.475\textwidth]{slides/Figures/standard_Linear_caterpillar.png}
  \includegraphics[width=0.475\textwidth]{slides/Figures/ww_Linear_caterpillar.png}
\end{frame}

%------------------------------------------------
%figures: Linear histogram and sum stats 
%-----------------------------------------------

\begin{frame}{Linear Weight: Performance Indicators} 

\centering

\includegraphics[width=.75\linewidth]{slides/Figures/Histrogram_Linear.png}

\scalebox{.8}{
    % latex table generated in R 3.6.1 by xtable 1.8-4 package
% Tue Jun 30 21:49:36 2020
\begin{tabular}{lrr}
  \hline
Statistic & Standard & Weighted \\ 
  \hline
Mean Squared Distance & 52.76 & 16.08 \\ 
  Mean Absolute Distance & 5.55 & 2.85 \\ 
  Correlation to Truth & 0.99 & 1.00 \\ 
   \hline
\end{tabular}

}
\end{frame}


%------------------------------------------------
%plot: rawlsian weight
%-----------------------------------------------

\begin{frame}{Rawlsian Weights}

\centering

\includegraphics[width=.85\linewidth]{slides/Figures/rawlsian_weight.png}

\end{frame}

%------------------------------------------------
%plot: rawlsian caterpillar plots 
%-----------------------------------------------

\begin{frame}{Rawlsian Weights: Welfare Added} 

\centering
 \includegraphics[width=0.475\textwidth]{slides/Figures/standard_rawlsian_caterpillar.png}
  \includegraphics[width=0.475\textwidth]{slides/Figures/ww_rawlsian_caterpillar.png}
\end{frame}


%------------------------------------------------
%figures: rawlsian histogram and sum stats 
%-----------------------------------------------

\begin{frame}{Rawlsian Weights: Performance Indicators} 

\centering

\includegraphics[width=.75\linewidth]{slides/Figures/Histrogram_rawlsian.png}

\scalebox{.8}{
    % latex table generated in R 3.6.1 by xtable 1.8-4 package
% Tue Jun 30 21:49:37 2020
\begin{tabular}{lrr}
  \hline
Statistic & Standard & Weighted \\ 
  \hline
Mean Squared Distance & 559.06 & 44.09 \\ 
  Mean Absolute Distance & 18.89 & 4.91 \\ 
  Correlation to Truth & 0.83 & 0.98 \\ 
   \hline
\end{tabular}

}
\end{frame}

%------------------------------------------------
%plot: Kernal weight
%-----------------------------------------------

\begin{frame}{Triangular Weights}

\centering

\includegraphics[width=.85\linewidth]{slides/Figures/Kernal_weight.png}

\end{frame}


%------------------------------------------------
%plot: Kernal caterpillar plots 
%-----------------------------------------------

\begin{frame}{VA Results: Triangular Weights} 

\centering
 \includegraphics[width=0.475\textwidth]{slides/Figures/standard_Kernal_caterpillar.png}
  \includegraphics[width=0.475\textwidth]{slides/Figures/ww_Kernal_caterpillar.png}
\end{frame}

%------------------------------------------------
%figures: Kernal histogram and sum stats 
%-----------------------------------------------

\begin{frame}{VA results: Triangular Weights} 

\centering

\includegraphics[width=.75\linewidth]{slides/Figures/Histrogram_Kernal.png}

\scalebox{.8}{
    % latex table generated in R 3.6.1 by xtable 1.8-4 package
% Tue Jun 30 21:49:37 2020
\begin{tabular}{lrr}
  \hline
Statistic & Standard & Weighted \\ 
  \hline
Mean Squared Distance & 364.29 & 35.17 \\ 
  Mean Absolute Distance & 15.38 & 4.24 \\ 
  Correlation to Truth & 0.88 & 0.99 \\ 
   \hline
\end{tabular}

}
\end{frame}


%------------------------------------------------------------------------------------------
%------------------------------------------------------------------------------------------
% stress test 
%------------------------------------------------------------------------------------------
%------------------------------------------------------------------------------------------
\section{Stress Test Weighted VAM}


%------------------------------------------------
%plot 
%-----------------------------------------------

\begin{frame}{WLS Measures Welfare Best When Pooling Across Years}
\vfill

\begin{figure}
    \includegraphics[width=.3\textwidth]{slides/Figures/Histrogram_rawlsian_Students_15.png}\hfill
    \includegraphics[width=.3\textwidth]{slides/Figures/Histrogram_rawlsian_Students_30.png}\hfill
    \includegraphics[width=.3\textwidth]{slides/Figures/Histrogram_rawlsian_Students_60.png}\hfill
    \\ \vspace{1em}
    \includegraphics[width=.3\textwidth]{slides/Figures/Histrogram_rawlsian_Students_100.png}\hfill
    \includegraphics[width=.3\textwidth]{slides/Figures/Histrogram_rawlsian_Students_140.png}\hfill
    \includegraphics[width=.3\textwidth]{slides/Figures/Histrogram_rawlsian_Students_200.png}\hfill
\end{figure}

\vfill

\end{frame}



%-------------------------------------------------------------
\begin{frame}{Measure Welfare Best When Pooling Across Years}
\vfill

\begin{figure}
    % Caterpillar Graph?
    \only<1>{
    \includegraphics[width=.45\textwidth]{slides/Figures/ww_rawlsian_caterpillar30.pdf}\hfill
    \includegraphics[width=.45\textwidth]{slides/Figures/ww_rawlsian_caterpillar60.pdf}}
    %\includegraphics[width=.3\textwidth]{slides/Figures/Histrogram_rawlsian_Students_60.png}\h
    %\\
    \only<2>{
    \includegraphics[width=.45\textwidth]{slides/Figures/ww_rawlsian_caterpillar100.pdf}\hfill
    \includegraphics[width=.45\textwidth]{slides/Figures/ww_rawlsian_caterpillar200.pdf}}
    %\includegraphics[width=.3\textwidth]{slides/Figures/Histrogram_rawlsian_Students_200.png}\hfill
\end{figure}
\vfill
\end{frame}


%-------------------------------------------------------------
% Did we want a frame showing that once there are a lot of students var(e_it) doesn't matter? We could just say it.


%------------------------------------------------------------------------------------------------------
% CONCLUDING SECTION 
%-------------------------------------------------------------------------------------------------------
\section{Feedback and Next Steps}

%------------------------------------------------
\begin{frame}[label=next]{Next Steps}

\begin{wideitemize}
    \item Explore quantile and kernel methods
    
    \hyperlink{methods}{\beamerbutton{Methods}}
    
    \item Other checks on method performance (e.g., nonrandom assignment)
    
    \item Apply for real data  (Discuss some options)
    
    \item All else equal we want to prioritize getting data with
    \begin{itemize}
        \item Tests every year (with minimal changes in testing structure)
        \item As many years as possible (since we want to explore pooling vs time-series)
        \item Information to control for unobservables \citep[e.g.,][]{schellenberg2020parents}
    \end{itemize}
    
    \item Extensions like teacher effort, teacher PPF, long term effects, policy counterfactuals
\end{wideitemize}

\end{frame}

%------------------------------------------------


\begin{frame}[c]
\centering
\Huge{\centerline{To Be Continued...}}
\normalsize njmather{\fontfamily{qag}\selectfont @}umich.edu \hspace{2em}
ricksmi{\fontfamily{qag}\selectfont @}umich.edu \hspace{2em} \normalsize teastmond{\fontfamily{qag}\selectfont @}ucsd.edu
\end{frame}

%------------------------------------------------


\begin{frame}
\frametitle{References}
\tiny
\bibliography{citations}
\end{frame}


%----------------------------------------------------------------------------------------
%----------------------------------------------------------------------------------------

\appendix
\setbeamertemplate{navigation symbols}{}


%----------------------------------------------------------------------------------------
%----------------------------------------------------------------------------------------


\begin{frame}[label=methods]{Other Methods: Overview}

\begin{wideitemize}

\item WLS has both pros and cons:
\begin{itemize}
    \item Intuitive and can estimate both VA heterogeneity and ``welfare added'' % for any welfare function
    \item The three-step procedure is unwieldy and will make inference challenging
    %    Especially given  multiple hypothesis tests we'd need to conduct at each point along the achievement distribution on the same data 
    \item Does it actually work econometrically...? 
    %(is seemingly unrelated regressions invalidated by reweighting in the second stage?)
\end{itemize}

\item We propose two other methods that could estimate heterogeneous VAM:
\begin{itemize}
    \item Conditional quantile regression
    \item Semiparametric index model
\end{itemize}

\end{wideitemize}

\end{frame}


%----------------------------------
% Method two: Quantile Regression
%--------------------------------------
\begin{frame}{Other Methods: Conditional Quantile Regression}

\begin{wideitemize}

    \item Conditional quantile regressions recover the ... \citep{}
    
    \item Simpler estimation procedure
    \begin{enumerate}
        \item Estimate the effect of teachers, $D_j$, at each conditional quantile ($p$)
        % There is a tension here about whether we want the Xs to be evaluated at the given quantile or not...
        %\item \color{gray}{or Use Frisch-Waugh-Lovell to partial out mean predictors then estimate effects of teachers on students at the (unconditional) $p$th quantile]?}
        \item Integrate over the effects using desired welfare weights
    \end{enumerate}

    \item Intuitive analogue to VAM methods that residualize out $X_i$ and run VAM on residuals
    
    \item Now the effect of $X_i$ is also being evaluated at the $p$th quantile (not the mean)...
    

\end{wideitemize}
\end{frame}

%----------------------------------
% Method Three: Kernel Regression
%------------------------------------

\begin{frame}{Other Methods: Semiparametric Index Model}

\begin{wideitemize}

    \item This method flows (most) directly from the SP motivation about weights
    
    \item  Nonparametrically identify teacher $j$'s effects over the achievement distribution
    \begin{itemize}
        \item Estimate $y_i = g_j(\beta X_i) +e_i$ using (or adapting?) Ichimura's method \citep{ichimura1993semiparametric}
        \item Integrate over the effects using desired welfare weights
    \end{itemize}
    
    \item Intuitive analogue: predict $\hat{y_i}$ then estimate $\mathbb{E}[y_i|\hat{y_i},j=j']$ for each teacher $j'$

    \item This will probably only work if we pool over years
    \item Bandwidth selection...? 
    %(Also a place to talk about measuring scores in pctle vs $\sigma$?)

\end{wideitemize}

\vfill
\begin{flushleft}

\hyperlink{next}{\beamerbutton{Back}}
\end{flushleft}
\end{frame}

\end{document}

